%\documentclass{...}

%modification history
% 9 jul 2011
% 29 nov 2011

\documentclass[11pt, letterpaper]{article}
\usepackage[margin=1in]{geometry}

% \documentclass[11pt, letterpaper]{article}
% 
% % ----- margins -----
% 
% \topmargin -1.5cm         % read Lamport p.163
% \oddsidemargin -0.04cm    % read Lamport p.163
% \evensidemargin -0.04cm   % same as oddsidemargin but for left-hand pages
% 
% % ----- texts -----
% 
% \textwidth 16.59cm
% \textheight 21.94cm
% 
% % ----- indendts and spacing -----
% 
% \parskip 0pt            	% spacing between paragraphs
% %\renewcommand{\baselinestretch}{1.5}	% uncomment for 1.5 spacing
% 
% \parindent 7mm		      % leading space for paragraphs between lines
% 
% % ----- page # -----
% 
% %\pagestyle{empty}         % uncomment if don't want page numbers





\usepackage{amsfonts, amsmath, amssymb, amsthm}
\usepackage{comment}
\usepackage{graphicx}
\usepackage{ifthen}
\usepackage{latexsym}
%\usepackage{times}
\usepackage[normalem]{ulem}

%modification history
% 2012
% may 3
% aug 11

%=== use the these packages if they are not already in use ===
%\usepackage{amsfonts, amsmath, amssymb, amsthm}
%=============================================================

%===== fonts =====
\def\ttt{\texttt}
%===== spacing =====

\def\extraspacing{\vspace{5mm} \noindent}
\def\figcapup{\vspace{-1mm}}
\def\figcapdown{\vspace{-0mm}}
\def\hgap{\textrm{\hspace{1mm}}}
\def\thmvgap{\vspace{0mm}}
\def\vgap{\vspace{2mm}}


%===== tabbing =====

\def\tab{\hspace{3mm}}
\def\tabpos{\hspace{4mm} \= \hspace{4mm} \= \hspace{4mm} \= \hspace{4mm} \=
\hspace{4mm} \= \hspace{4mm} \= \hspace{4mm} \= \hspace{4mm} \= \hspace{4mm}
\kill}

%===== blocks =====

\newtheorem{theorem}{Theorem}
\newtheorem{lemma}{Lemma}
\newtheorem{corollary}{Corollary}
\newtheorem{proposition}{Proposition}
\newtheorem{definition}{Definition}
\newtheorem{problem}{Problem}

%===== math macros =====

\def\bm{\boldmath}
%\def\defeq{\stackrel{\textrm{\tiny{def}}}{=}}
\def\defeq{:=}
\def\eps{\epsilon}
\def\fr{\frac}
\def\-{\mbox{-}}
\def\inte{\mathbb{N}}
\def\ovline{\overline}
\def\real{\mathbb{R}}

\def\lc{\lceil}
\def\lf{\lfloor}
\def\rc{\rceil}
\def\rf{\rfloor}

\def\nn{\nonumber}

\def\Pr{\mathbf{Pr}}
\def\expt{\mathbf{E}}
\def\var{\mathbf{var}}

\def\*{\star}

\DeclareMathOperator*{\argmin}{arg\,min}
\DeclareMathOperator*{\polylg}{polylg}
\DeclareMathOperator*{\polylog}{polylog}
\DeclareMathOperator*{\intr}{\cap}

%===== misc =====

\def\done{\hspace*{\fill} $\framebox[2mm]{}$}	% end of proof
%\def\done{\hspace*{\fill} $\Box$}	% end of proof

%===== coloring =====

\newcommand{\red}[1]{\textcolor{red}{#1}}


\def\dom{\prec}
\def\T{\mathcal{T}}

\newboolean{solver}\setboolean{solver}{true}
%\newboolean{solver}\setboolean{solver}{false}
\ifthenelse{\boolean{solver}}{\includecomment{sol}}{\excludecomment{sol}}

\begin{document}

\section*{SC3020-CE4301-CZ4031: Tutorial 7}
%Prepared by Yufei Tao \\

\begin{center}
    \uline{Classroom Discussion}
\end{center}

Let $R_1(A, B)$ be a relation with attributes $A$ and $B$, and $R_2(A, C)$ be a relation with attributes $A$ and $C$. We assume that there is a B-tree on $R_2$ indexing the attribute $A$. Consider the following four algorithms for computing $R_1 \bowtie R_2$.

\myitems{
    \item Block-based nested loop (BNL)
    \item Sort Join (SJ)
    \item Hash Join (HJ)
    \item Index-based nested loop (INL). This algorithm works as follows: for each tuple $t_1 \in R_1$, probe the B-tree on $R_2$ to find all the tuples $t_2 \in R_2$ satisfying the condition $t_1.A = t_2.A$.
}

\noindent The {\em query optimizer} is responsible for selecting the {\bf best query plan} whose {\em predicted I/O cost} is the lowest. In the following problems, we will gain an idea of how the query plan is chosen.

\extraspacing {\bf Problem 1.} Suppose we know:
\myitems{
    \item $R_1$ has $10^5$ tuples, and $R_2$ has $10^7$ tuples.
    \item Each block can store 10 tuples of $R_1$ or $R_2$. Hence, $R_1$ is stored in $B(R_1) = 10^4$ blocks, and $R_2$ in $B(R_2) = 10^6$.
    \item Neither $R_1$ nor $R_2$ is sorted on $A$.
    \item The memory has $M = 6001$ blocks.
    \item The no skew assumption holds for SJ, and the good hashing assumption holds for HJ.
    \item The B-tree on $R_2$ has 3 levels.
    \item For each tuple $t_1 \in R_1$, there are 55 tuples $t_2$ in $R_2$ satisfying the condition $t_1.A = t_2.A$.
}
If you are the query optimizer, how would you compute $R_1 \bowtie R_2$?

\begin{sol}
\extraspacing {\bf Solution.} We make the choice by predicting the I/O cost of each algorithm.

\myitems{
    \item BNL\\
    I/O cost $= B(R_1) + \lc B(R_1)/(M-1) \rc \cdot B(R_2) = 10^4 + \lc \fr{10^4}{6000} \rc \cdot 10^6 = 2010000$.

    \item SJ\\
    I/O cost $= 5 (B(R_1) + B(R_2)) = 5050000$.

    \item HJ\\
    I/O cost $= 3 (B(R_1) + B(R_2)) = 3030000$.

    \item INL\\
    For each tuple $t_1 \in R_1$, probing the B-tree of $R_2$ (which is un-clustered) requires one I/O per level\footnote{In general, more than one I/O may be needed to access the leaf level of an un-clustered B-tree. However, in query optimization, it is often acceptable to assume that only one I/O is paid at each level. } and then 55 I/Os to retrieve the 55 tuples $t_2 \in R_2$ satisfying $t_1.A = t_2.A$ (one I/O per tuple). This is $58$ I/Os per tuple in $R_1$. As $R_1$ has $10^5$ tuples, the total I/O cost of B-tree probing is $58 \times 10^5 = 5800000$. In addition, reading the tuples of $R_1$ requires another $B(R_1) = 10^4$ blocks. Thus, overall, INL performs $5810000$ I/Os.
}

The best approach is to perform BNL.
\end{sol}


\extraspacing {\bf Problem 2.} Suppose we know:
\myitems{
    \item $R_1$ has $10^5$ tuples, and $R_2$ has $10^7$ tuples.
    \item Each block can store 10 tuples of $R_1$ or $R_2$. Hence, $R_1$ is stored in $B(R_1) = 10^4$ blocks, and $R_2$ in $B(R_2) = 10^6$.
    \item Neither $R_1$ nor $R_2$ is sorted on $A$.
    \item The memory has $M = 2001$ blocks.
    \item The no skew assumption holds for SJ, and the good hashing assumption holds for HJ.
    \item The B-tree on $R_2$ has 3 levels.
    \item For each tuple $t_1 \in R_1$, there are 15 tuples $t_2$ in $R_2$ satisfying the condition $t_1.A = t_2.A$.
}
If you are the query optimizer, how would you compute $R_1 \bowtie R_2$?

\begin{sol}
\extraspacing {\bf Solution.} We make the choice by predicting the I/O cost of each algorithm.

\myitems{
    \item BNL\\
    I/O cost $= B(R_1) + \lc B(R_1)/(M-1) \rc \cdot B(R_2) = 10^4 + \lc \fr{10^4}{2000} \rc \cdot 10^6 = 5010000$.

    \item SJ\\
    I/O cost $= 5 (B(R_1) + B(R_2)) = 5050000$.

    \item HJ\\
    I/O cost $= 3 (B(R_1) + B(R_2)) = 3030000$.

    \item INL\\
    I/O cost $= 5810000$ I/Os as analyzed in Problem 1.
}

The best approach is to perform HJ.
\end{sol}

\extraspacing {\bf Problem 3.} Suppose we know:
\myitems{
    \item $R_1$ has $10^5$ tuples, and $R_2$ has $10^7$ tuples.
    \item Each block can store 10 tuples of $R_1$ or $R_2$. Hence, $R_1$ is stored in $B(R_1) = 10^4$ blocks, and $R_2$ in $B(R_2) = 10^6$.
    \item {\bf $R_1$ is not sorted on $A$, but $R_2$ is sorted on $A$.}
    \item The memory has $M = 2001$ blocks.
    \item The no skew assumption holds for SJ, and the good hashing assumption holds for HJ.
    \item The B-tree on $R_2$ has 3 levels.
    \item For each tuple $t_1 \in R_1$, there are 15 tuples $t_2$ in $R_2$ satisfying the condition $t_1.A = t_2.A$.
}
If you are the query optimizer, how would you compute $R_1 \bowtie R_2$?


\begin{sol}
\extraspacing {\bf Solution.} We make the choice by predicting the I/O cost of each algorithm.

\myitems{
    \item BNL\\
    I/O cost $= 5010000$ as analyzed in Problem 2.

    \item SJ\\
    As $R_2$ is already sorted on $A$, SJ only needs to sort $R_1$, which costs $4 \cdot B(R_1)$. After that, one synchronous scan of $R_1$ and $R_2$ finds the result of $R_1 \bowtie R_2$. Hence, the I/O cost is $= 5 \cdot B(R_1) + B(R_2) = 1050000$.

    \item HJ\\
    I/O cost $= 3 (B(R_1) + B(R_2)) = 3030000$.

    \item INL\\
    Note that the B-tree of $R_2$ is clustered in this scenario!
    For each tuple $t_1 \in R_1$, probing the B-tree of $R_2$ requires one I/O per level and then $\lc 55/10 \rc = 6$ I/Os to retrieve the 100 tuples $t_2 \in R_2$ satisfying $t_1.A = t_2.A$ (one I/O per tuple). This is $9$ I/Os per tuple in $R_1$. As $R_1$ has $10^5$ tuples, the total I/O cost of B-tree probing is $9 \times 10^5 = 900000$. In addition, reading the tuples of $R_1$ requires another $B(R_1) = 10^4$ blocks. Thus, overall, INL performs $910000$ I/Os.
}

The best approach is to perform INL.
\end{sol}

\extraspacing {\bf Problem 4.} Suppose we know:
\myitems{
    \item $R_1$ has $10^5$ tuples, and $R_2$ has $10^7$ tuples.
    \item Each block can store 10 tuples of $R_1$ or $R_2$. Hence, $R_1$ is stored in $B(R_1) = 10^4$ blocks, and $R_2$ in $B(R_2) = 10^6$.
    \item $R_1$ is not sorted on $A$, but $R_2$ is sorted on $A$.
    \item The memory has $M = 2002$ blocks.
    \item The no skew assumption holds for SJ, and the good hashing assumption holds for HJ.
    \item The B-tree on $R_2$ has 3 levels.
    \item For each tuple $t_1 \in R_1$, there are 15 tuples $t_2$ in $R_2$ satisfying the condition $t_1.A = t_2.A$.
}
We want to compute $R_1 \bowtie R_2$, sort the result on attribute $A$, and store the sorted list on disk. You can assume that each block can store 7 tuples in the join result. If you are the query optimizer, how would you process the query?


\begin{sol}
\extraspacing {\bf Solution.} We make the choice by predicting the I/O cost of each algorithm.

\myitems{
    %\item BNL\\
    %I/O cost $= 5010000$ as analyzed in Problem 2.

    \item SJ\\
    The join result contains $10^5 \times 15 = 1500000$ tuples, which fit in $\lc 1500000/7 \rc = 214286$ I/Os. SJ performs 1050000 I/Os as analyzed in Problem 3, plus $214286$ I/Os for writing the output --- note that the output is already sorted by $A$. The total I/O cost is 1124286.

    \item INL\\
    INL performs 910000 I/Os as analyzed in Problem 3, plus 214286 I/Os for writing the output. After that, it is still necessary to sort the join result on $A$, the cost of which is $4 \times 214286 = 857144$ I/Os (external sort). Hence, the total I/O cost is 1981430.
}
We omit the cost analysis of BNL and HJ (which must perform more I/Os than in Problem 3). The best approach is to perform SJ.
\end{sol}



\end{document}
