%\documentclass{...}

%modification history
% 9 jul 2011
% 29 nov 2011

\documentclass[11pt, letterpaper]{article}
\usepackage[margin=1in]{geometry}

% \documentclass[11pt, letterpaper]{article}
% 
% % ----- margins -----
% 
% \topmargin -1.5cm         % read Lamport p.163
% \oddsidemargin -0.04cm    % read Lamport p.163
% \evensidemargin -0.04cm   % same as oddsidemargin but for left-hand pages
% 
% % ----- texts -----
% 
% \textwidth 16.59cm
% \textheight 21.94cm
% 
% % ----- indendts and spacing -----
% 
% \parskip 0pt            	% spacing between paragraphs
% %\renewcommand{\baselinestretch}{1.5}	% uncomment for 1.5 spacing
% 
% \parindent 7mm		      % leading space for paragraphs between lines
% 
% % ----- page # -----
% 
% %\pagestyle{empty}         % uncomment if don't want page numbers





\usepackage{amsfonts, amsmath, amssymb, amsthm}
\usepackage{comment}
\usepackage{graphicx}
\usepackage{ifthen}
\usepackage{latexsym}
%\usepackage{times}
\usepackage[normalem]{ulem}

%modification history
% 2012
% may 3
% aug 11

%=== use the these packages if they are not already in use ===
%\usepackage{amsfonts, amsmath, amssymb, amsthm}
%=============================================================

%===== fonts =====
\def\ttt{\texttt}
%===== spacing =====

\def\extraspacing{\vspace{5mm} \noindent}
\def\figcapup{\vspace{-1mm}}
\def\figcapdown{\vspace{-0mm}}
\def\hgap{\textrm{\hspace{1mm}}}
\def\thmvgap{\vspace{0mm}}
\def\vgap{\vspace{2mm}}


%===== tabbing =====

\def\tab{\hspace{3mm}}
\def\tabpos{\hspace{4mm} \= \hspace{4mm} \= \hspace{4mm} \= \hspace{4mm} \=
\hspace{4mm} \= \hspace{4mm} \= \hspace{4mm} \= \hspace{4mm} \= \hspace{4mm}
\kill}

%===== blocks =====

\newtheorem{theorem}{Theorem}
\newtheorem{lemma}{Lemma}
\newtheorem{corollary}{Corollary}
\newtheorem{proposition}{Proposition}
\newtheorem{definition}{Definition}
\newtheorem{problem}{Problem}

%===== math macros =====

\def\bm{\boldmath}
%\def\defeq{\stackrel{\textrm{\tiny{def}}}{=}}
\def\defeq{:=}
\def\eps{\epsilon}
\def\fr{\frac}
\def\-{\mbox{-}}
\def\inte{\mathbb{N}}
\def\ovline{\overline}
\def\real{\mathbb{R}}

\def\lc{\lceil}
\def\lf{\lfloor}
\def\rc{\rceil}
\def\rf{\rfloor}

\def\nn{\nonumber}

\def\Pr{\mathbf{Pr}}
\def\expt{\mathbf{E}}
\def\var{\mathbf{var}}

\def\*{\star}

\DeclareMathOperator*{\argmin}{arg\,min}
\DeclareMathOperator*{\polylg}{polylg}
\DeclareMathOperator*{\polylog}{polylog}
\DeclareMathOperator*{\intr}{\cap}

%===== misc =====

\def\done{\hspace*{\fill} $\framebox[2mm]{}$}	% end of proof
%\def\done{\hspace*{\fill} $\Box$}	% end of proof

%===== coloring =====

\newcommand{\red}[1]{\textcolor{red}{#1}}


\def\extraspacing{\vspace{5mm} \noindent}

\def\best{\mathit{best}}
\def\size{\mathit{size}}

\newboolean{solver}\setboolean{solver}{true}
%\newboolean{solver}\setboolean{solver}{false}
\ifthenelse{\boolean{solver}}{\includecomment{sol}}{\excludecomment{sol}}

\begin{document}

\section*{SC3020-CE4301-CZ4031: Tutorial 11}
Prepared by Yufei Tao \\

\begin{center}
    \uline{Classroom Discussion}
\end{center}

\extraspacing {\bf Problem 1.} Let $R(X, Y)$ be a relation with attributes $X$ and $Y$. We enforce the \ttt{READ COMMITTED} isolation level with the MVCC protocol. Here is the history of a tuple $t$ in the relation:
\begin{center}
    ((10, 10), $T_1$, $T_3$), ((20, 20), $T_3$, $T_2$), ((30, 30), $T_2$, \ttt{NULL}).
\end{center}
Answer the following questions:
\myitems{
    \item What is the status of $T_1$ (i.e., committed, running, or aborted)? How about $T_2$ and $T_3$?
    \item If a transaction $T$ issues a READ($t$), what are the values read in each of the situations below?
    \myitems{
        \item [(a)] $T = T_2$;
        \item [(b)] $T = T_4$, and $T_2$ is still running;
        \item [(c)] $T = T_4$, and $T_2$ has committed.
    }
}


\begin{sol}
\extraspacing {\bf Solution.} Situation (a): $(30, 30)$. Situation (b): $(20, 20)$. Situation (c): $(30, 30)$.

\end{sol}

\extraspacing {\bf Problem 2.} Suppose that the database attempts to execute the following instructions in the order shown using the MVCC protocol.

\begin{center}
\begin{tabular}{c|c|c|c|c}
 & $T_1$ (ID 201) & $T_2$ (ID 202) & $T_3$ (ID 203) & $T_4$ (ID 204) \\
\hline
1  & LOCK($t$)       &                &                &                \\
2  & WRITE($t$, 10)  &                &                &                \\
3  & COMMIT           &                &                &                \\
4  &                  & READ($t$)      &                &                \\
5  &                  & LOCK($t$)      &                &                \\
6  &                  & WRITE($t$, 15) &                &                \\
7  &                  &                & LOCK($t$)      &                \\
8  &                  &                & WRITE($t$, 18) &                \\
9  &                  &                & ABORT           &                \\
10 &                  &                &                & READ($t$)      \\
11 &                  &                &                & LOCK($t$)      \\
12 &                  &                &                & WRITE($t$, 25) \\
13 &                  &                &                & COMMIT          \\
14 &                  & COMMIT         &                &                \\
\end{tabular}
\end{center}

Let us assume that $t$ is inserted to the database by the WRITE instruction of $T_1$. Answer the following questions.
\myitems{
    \item [(a)] Show the schedule produced, and give the history of $t$ after every instruction.
    \item [(b)] What is the value returned for each READ instruction?
    \item [(c)] Is the schedule serializable?
}

\begin{sol}
    \extraspacing {\bf Solution.} (a)
    \begin{center}
        \begin{tabular}{c|c}
            instruction & remark \\
            \hline
            $T_1$: LOCK($t$) & \\
            $T_1$: WRITE($t$, 10) & history = (10, 201, NULL) \\
            $T_1$: COMMIT &  \\
            $T_2$: READ($t$) & \\
            $T_2$: LOCK($t$)              & \\
            $T_2$: WRITE($t$, 15)         & history = (10, 201, 202), (15, 202, NULL) \\
            $T_3$: LOCK($t$)              & denied; $T_3$ put on hold \\
            $T_4$: READ($t$)              & \\
            $T_4$: LOCK($t$)              & denied; $T_4$ put on hold \\
            $T_2$: COMMIT                 & $T_2$ releases lock \\
            $T_3$: LOCK($t$)              & granted \\
            $T_3$: WRITE($t$, 18)         & history = (10, 201, 202), (15, 202, 203), (18, 203, NULL) \\
            $T_3$: ABORT                  & history reverts to (10, 201, 202), (15, 202, NULL); $T_3$ released lock \\
            $T_4$: LOCK($t$)              & granted \\
            $T_4$: WRITE($t$, 25)         & history = (10, 201, 202), (15, 202, 204), (25, 204, NULL) \\
            $T_4$: COMMIT                 &
        \end{tabular}
    \end{center}

    (b) Both READ operations return 10.

    (c) To analyze conflict serializability, let us focus on the schedule's I/O instructions:

    \begin{center}
        \begin{tabular}{c|c}
            instruction & remark \\
            \hline
            1 & $T_1$: WRITE($t$, 10) \\
            2 & $T_2$: READ($t$) \\
            3 & $T_2$: WRITE($t$, 15) \\
            4 & $T_4$: READ($t$) \\
            5 & $T_3$: WRITE($t$, 18) \\
            6 & $T_4$: WRITE($t$, 25)
        \end{tabular}
    \end{center}

    Instructions 4 and 5 indicate that the precedence graph has an edge from $T_4$ to $T_3$. On the other hand, instructions 5 and 6 indicate that the precedence graph has an edge from $T_3$ to $T_4$. Therefore, the schedule is not conflict serializable.

\end{sol}

\extraspacing {\bf Problem 3.} Give a schedule that can produce the history in Problem 1. In your schedule, $T_1$ must be the first transaction to start, $T_2$ is the second, and $T_3$ is the last. It suffices to show the following instructions: START, WRITE, and COMMIT.

\begin{sol}
\extraspacing {\bf Solution.}

\begin{center}
 \begin{tabular}{c|c}
 no & instruction \\
 \hline
  1 & $T_1:$ START \\
  2 & $T_1:$ WRITE$(t, (10, 10))$ \\
  3 & $T_1:$ COMMIT \\
  4 & $T_2:$ START \\
  5 & $T_3:$ START \\
  6 & $T_3:$ WRITE$(t, (20, 20))$ \\
  7 & $T_3:$ COMMIT \\
  8 & $T_2:$ WRITE$(t, (30, 30))$ \\
  9 & $T_2:$ COMMIT
 \end{tabular}
\end{center}

\end{sol}

\extraspacing {\bf Problem 4.} Construct a schedule under MVCC that causes a deadlock.

\begin{sol}
    \extraspacing {\bf Solution.}

\begin{center}
    \begin{tabular}{c|c}
    no & instruction \\
    \hline
    1 & $T_1:$ START \\
    2 & $T_1:$ LOCK$(t_1)$ \\
    3 & $T_2:$ START \\
    4 & $T_2:$ LOCK$(t_2)$ \\
    5 & $T_1:$ LOCK$(t_2)$ \\
    6 & $T_2:$ LOCK$(t_1)$
    \end{tabular}
    \end{center}
\end{sol}


\extraspacing

\begin{center}
    \uline{Critical Thinking}
\end{center}


\end{document}
