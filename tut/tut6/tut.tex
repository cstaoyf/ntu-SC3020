%\documentclass{...}

%modification history
% 9 jul 2011
% 29 nov 2011

\documentclass[11pt, letterpaper]{article}
\usepackage[margin=1in]{geometry}

% \documentclass[11pt, letterpaper]{article}
% 
% % ----- margins -----
% 
% \topmargin -1.5cm         % read Lamport p.163
% \oddsidemargin -0.04cm    % read Lamport p.163
% \evensidemargin -0.04cm   % same as oddsidemargin but for left-hand pages
% 
% % ----- texts -----
% 
% \textwidth 16.59cm
% \textheight 21.94cm
% 
% % ----- indendts and spacing -----
% 
% \parskip 0pt            	% spacing between paragraphs
% %\renewcommand{\baselinestretch}{1.5}	% uncomment for 1.5 spacing
% 
% \parindent 7mm		      % leading space for paragraphs between lines
% 
% % ----- page # -----
% 
% %\pagestyle{empty}         % uncomment if don't want page numbers





\usepackage{amsfonts, amsmath, amssymb, amsthm}
\usepackage{comment}
\usepackage{graphicx}
\usepackage{ifthen}
\usepackage{latexsym}
%\usepackage{times}

%modification history
% 2012
% may 3
% aug 11

%=== use the these packages if they are not already in use ===
%\usepackage{amsfonts, amsmath, amssymb, amsthm}
%=============================================================

%===== fonts =====
\def\ttt{\texttt}
%===== spacing =====

\def\extraspacing{\vspace{5mm} \noindent}
\def\figcapup{\vspace{-1mm}}
\def\figcapdown{\vspace{-0mm}}
\def\hgap{\textrm{\hspace{1mm}}}
\def\thmvgap{\vspace{0mm}}
\def\vgap{\vspace{2mm}}


%===== tabbing =====

\def\tab{\hspace{3mm}}
\def\tabpos{\hspace{4mm} \= \hspace{4mm} \= \hspace{4mm} \= \hspace{4mm} \=
\hspace{4mm} \= \hspace{4mm} \= \hspace{4mm} \= \hspace{4mm} \= \hspace{4mm}
\kill}

%===== blocks =====

\newtheorem{theorem}{Theorem}
\newtheorem{lemma}{Lemma}
\newtheorem{corollary}{Corollary}
\newtheorem{proposition}{Proposition}
\newtheorem{definition}{Definition}
\newtheorem{problem}{Problem}

%===== math macros =====

\def\bm{\boldmath}
%\def\defeq{\stackrel{\textrm{\tiny{def}}}{=}}
\def\defeq{:=}
\def\eps{\epsilon}
\def\fr{\frac}
\def\-{\mbox{-}}
\def\inte{\mathbb{N}}
\def\ovline{\overline}
\def\real{\mathbb{R}}

\def\lc{\lceil}
\def\lf{\lfloor}
\def\rc{\rceil}
\def\rf{\rfloor}

\def\nn{\nonumber}

\def\Pr{\mathbf{Pr}}
\def\expt{\mathbf{E}}
\def\var{\mathbf{var}}

\def\*{\star}

\DeclareMathOperator*{\argmin}{arg\,min}
\DeclareMathOperator*{\polylg}{polylg}
\DeclareMathOperator*{\polylog}{polylog}
\DeclareMathOperator*{\intr}{\cap}

%===== misc =====

\def\done{\hspace*{\fill} $\framebox[2mm]{}$}	% end of proof
%\def\done{\hspace*{\fill} $\Box$}	% end of proof

%===== coloring =====

\newcommand{\red}[1]{\textcolor{red}{#1}}


\def\dom{\prec}
\def\T{\mathcal{T}}

%\newboolean{solver}\setboolean{solver}{true}
\newboolean{solver}\setboolean{solver}{false}
\ifthenelse{\boolean{solver}}{\includecomment{sol}}{\excludecomment{sol}}

\begin{document}

\section*{CSCI2100: Regular Exercise Set 1}
Prepared by Yufei Tao \\


\extraspacing {\bf Problem 1.} Let $x$ be a real value. Define $\lf x \rf$ to be the largest integer that does not exceed $x$. For example, $\lf 2.5 \rf = 2$, whereas $\lf 3 \rf = 3$. 

\vgap

Suppose that you are given an integer $n \ge 2$ in (a register of) the CPU. Write an algorithm to compute the value of $\lf \log_2 n \rf$ in no more than $100 \log_2 n$ time. 

\begin{sol}
\extraspacing {\bf Solution.} We will generate $2^1, 2^2, 2^3, ...$ until finding the smallest $i$ such that $2^i > n$. Initially, set $i = 1$ and $p = 2$. Repeat the following until $p > n$: 
\begin{itemize} 
 \item Increase $i$ by 1. 
 \item Increase $p$ by a factor of 2.  
\end{itemize}
Finally, return $i - 1$ as the answer. It is clear that $i$ is increased no more than $1 + \log_2 n$ times. For each value of $i$, performing the aforementioned steps can be easily implemented in less than 10 atomic operations. The total cost is therefore at most $10 (1 + \log_2 n)$ which is less than $100 \log_2 n$ for all $n \ge 2$.
\end{sol}

\extraspacing {\bf Problem 2.} The following figure shows an input to the dictionary search problem. 

\begin{center} 
    \includegraphics[height=30mm]{./artwork/bin_srch}
\end{center}

\noindent Describe how binary search works using the input.

\begin{sol} 
\extraspacing {\bf Solution.} First, compare 35 to the 8th element 52 in memory. Since $35 < 52$, we can now focus on the first 7 elements. Next, compare 35 to the 4th element 26. Since $35 > 26$, we only need to consider the 5th, 6th, and 7th elements. Then, the algorithm compares 35 first to the 6th, and then to the 5th, before returning the answer ``no''.

\extraspacing {\bf \underline{Remark.}} Binary search looks for the ``middle element'' among the subset of elements that are still in consideration. However, what it means by the ``middle'' can be subjective. In the above, we adopted the following convention. Suppose that $t$ elements still remain. If $t$ is an odd number, then the middle element is the one that stands strictly in the middle (such as 26 in the above example). If $t$ is even, then the middle element is the $(t/2)$-th (such as 52 in the above example). This is---purposely---different from what was shown in the lecture notes. In quizzes/exams, you are free to adopt any conventions. 
\end{sol}


\extraspacing {\bf Problem 3 (Predecessor Search).} Let us first define the notion of {\em predecessor}. Let $S$ be a set of integers. Given an integer $v$, the {\em predecessor} of $v$ in $S$ is the largest integer in $S$ that is at most $v$. For example, suppose $S = \{3, 14, 15, 26, 32, 40\}$. The predecessor of $25$ is 15, while that of 26 is 26. 

\vgap

Consider the following problem. You are given a set $S$ of $n$ integers, which are stored at memory cells 1, 2, ..., $n$ in ascending order. The value of $n$ is given in the CPU, and so is an integer $v$. The following shows an example with $n = 16$ and $v = 35$. 

\begin{center} 
    \includegraphics[height=30mm]{./artwork/bin_srch}
\end{center}

Describe an algorithm to find the predecessor of $v$. Your algorithm should have running time at most $100 + 100 \log_2 n$. 

\begin{sol} 
\extraspacing {\bf Solution.} First perform binary search on $S$ using $v$. Let $x$ be the last element that the algorithm compared $v$ to. If $x = v$, then $x$ is the predecessor of $v$. If $x < v$, then $x$ is the predecessor of $v$. Finally, if $x > v$, then the predecessor of $v$ is the element immediately before $x$---in the special case where $x$ is already the smallest element in $S$, then $v$ has no predecessor in $S$. 

\vgap 

Binary search takes no more than $7 \log_2 n$ atomic operations. Clearly, the algorithm finishes in less than 10 atomic operations after binary search. The overall cost is therefore less than $10 + 7\log_2 n < 100 \log_2 n$. 
\end{sol}


\extraspacing {\bf Problem 4 (Prefix Counting).} Consider the following problem. You are given a set $S$ of $n$ integers, which are stored at memory cells 1, 2, ..., $n$ in ascending order. The value of $n$ is given in the CPU, and so is an integer $v$. The following shows an example with $n = 16$ and $v = 35$. 

\begin{center} 
    \includegraphics[height=30mm]{./artwork/bin_srch}
\end{center}

\noindent Describe an algorithm to find the number of integers in $S$ that are at most $v$. In the above example, for instance, you should return 5. Your algorithm should have running time at most $100 + 100 \log_2 n$. 

\begin{sol} 
\extraspacing {\bf Solution.} Notice that our predecessor search algorithm in Problem 4 not only finds the predecessor $x$ of $v$, but also the address $a$ of the memory cell where $x$ is stored. To solve the prefix counting problem, first find $x$. If $x$ does not exist, return 0. Otherwise, return $a$. The cost is clearly no more than $100 \log_2 n$.  
\end{sol}

\extraspacing {\bf Problem 5 (The 3-Sum Problem).} Consider the following problem. The input $S$ consists of $n$ integers, which are given at memory cells 1, 2, ..., $n$, arranged in ascending order. The value of $n$ is given in the CPU. So is a value $v$. The following shows an example with $n = 16$ and $v = 150$.

\begin{center} 
    \includegraphics[height=30mm]{./artwork/3_sum}
\end{center}

\noindent Describe an algorithm to determine whether $S$ has 3 numbers that sum up to $v$. In the above example, the answer is ``yes'' because $150 = 40 + 45 + 65$. Your algorithm should have running time at most $100 + 100 \cdot n^2 \log_2 n$.

\begin{sol} 
\extraspacing {\bf Solution.} The tutorial slides described how to solve the following ``2-sum'' problem. There, we are given a set $S$ of $n$ integers as in the 3-Sum problem, the value of $n$, and also an integer $v$. The goal is to determine whether $S$ contains two numbers that add up to $v$. The tutorial slides presented two algorithms for solving this problem. The first one has running time less than $7 n \log_2 n$. 

\vgap 

Observe that we can settle the 3-sum problem by solving $n$ instances of the 2-sum problem. Specifically, for each $i \in [1, n]$, let $x$ be the $i$-th integer from $S$. We invoke an algorithm for solving the 2-sum problem where we ask whether $S$ contains 2 elements (other than $x$) that sum up to $v - x$. If so, then we return ``yes'' immediately to the 3-sum problem. Otherwise, we attempt the next $i$. After all the values of $i$ have been attempted, we return ``no''. 

\vgap 

It is easy to implement the above strategy in cost less than $100 \cdot n^2 \log_2 n$. 
\end{sol}


% \extraspacing {\bf Problem 6.} Still the same problem as above, but improve the running time of your algorithm to at most $100 \cdot n^2$. 
% 
% \begin{sol} 
% \extraspacing {\bf Solution.} The second algorithm given in the tutorial slides for the 2-sum problem has running time less than $6n$. Plugging in this algorithm in our solution framework for Problem 5 gives a new algorithm with running time less than $100 \cdot n^2$.
% \end{sol}



%\bibliographystyle{abbrv}
%\bibliography{../ref}

\end{document}
