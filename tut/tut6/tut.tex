%\documentclass{...}

%modification history
% 9 jul 2011
% 29 nov 2011

\documentclass[11pt, letterpaper]{article}
\usepackage[margin=1in]{geometry}

% \documentclass[11pt, letterpaper]{article}
% 
% % ----- margins -----
% 
% \topmargin -1.5cm         % read Lamport p.163
% \oddsidemargin -0.04cm    % read Lamport p.163
% \evensidemargin -0.04cm   % same as oddsidemargin but for left-hand pages
% 
% % ----- texts -----
% 
% \textwidth 16.59cm
% \textheight 21.94cm
% 
% % ----- indendts and spacing -----
% 
% \parskip 0pt            	% spacing between paragraphs
% %\renewcommand{\baselinestretch}{1.5}	% uncomment for 1.5 spacing
% 
% \parindent 7mm		      % leading space for paragraphs between lines
% 
% % ----- page # -----
% 
% %\pagestyle{empty}         % uncomment if don't want page numbers





\usepackage{amsfonts, amsmath, amssymb, amsthm}
\usepackage{comment}
\usepackage{graphicx}
\usepackage{ifthen}
\usepackage{latexsym}
%\usepackage{times}
\usepackage[normalem]{ulem}

%modification history
% 2012
% may 3
% aug 11

%=== use the these packages if they are not already in use ===
%\usepackage{amsfonts, amsmath, amssymb, amsthm}
%=============================================================

%===== fonts =====
\def\ttt{\texttt}
%===== spacing =====

\def\extraspacing{\vspace{5mm} \noindent}
\def\figcapup{\vspace{-1mm}}
\def\figcapdown{\vspace{-0mm}}
\def\hgap{\textrm{\hspace{1mm}}}
\def\thmvgap{\vspace{0mm}}
\def\vgap{\vspace{2mm}}


%===== tabbing =====

\def\tab{\hspace{3mm}}
\def\tabpos{\hspace{4mm} \= \hspace{4mm} \= \hspace{4mm} \= \hspace{4mm} \=
\hspace{4mm} \= \hspace{4mm} \= \hspace{4mm} \= \hspace{4mm} \= \hspace{4mm}
\kill}

%===== blocks =====

\newtheorem{theorem}{Theorem}
\newtheorem{lemma}{Lemma}
\newtheorem{corollary}{Corollary}
\newtheorem{proposition}{Proposition}
\newtheorem{definition}{Definition}
\newtheorem{problem}{Problem}

%===== math macros =====

\def\bm{\boldmath}
%\def\defeq{\stackrel{\textrm{\tiny{def}}}{=}}
\def\defeq{:=}
\def\eps{\epsilon}
\def\fr{\frac}
\def\-{\mbox{-}}
\def\inte{\mathbb{N}}
\def\ovline{\overline}
\def\real{\mathbb{R}}

\def\lc{\lceil}
\def\lf{\lfloor}
\def\rc{\rceil}
\def\rf{\rfloor}

\def\nn{\nonumber}

\def\Pr{\mathbf{Pr}}
\def\expt{\mathbf{E}}
\def\var{\mathbf{var}}

\def\*{\star}

\DeclareMathOperator*{\argmin}{arg\,min}
\DeclareMathOperator*{\polylg}{polylg}
\DeclareMathOperator*{\polylog}{polylog}
\DeclareMathOperator*{\intr}{\cap}

%===== misc =====

\def\done{\hspace*{\fill} $\framebox[2mm]{}$}	% end of proof
%\def\done{\hspace*{\fill} $\Box$}	% end of proof

%===== coloring =====

\newcommand{\red}[1]{\textcolor{red}{#1}}


\def\dom{\prec}
\def\T{\mathcal{T}}

%\newboolean{solver}\setboolean{solver}{true}
\newboolean{solver}\setboolean{solver}{false}
\ifthenelse{\boolean{solver}}{\includecomment{sol}}{\excludecomment{sol}}

\begin{document}

\section*{SC3020-CZ4031: Tutorial 6}
%Prepared by Yufei Tao \\

\begin{center}
    \uline{Classroom Discussion}
\end{center}


\extraspacing {\bf Problem 1.} Suppose that we have three sets of integers, denoted as $S_1, S_2$, and $S_3$, respectively. Each set is sorted and given to you in a file. Assume that each disk block can hold 2 integers, and the memory has 4 blocks. The content of $S_1, S_2$, and $S_3$ is shown below.

\vgap

\noindent $S_1:$ $\uline{[1, 10]}, \uline{[20, 30]}, \uline{[40, 50]}$ \\
$S_2:$ $\uline{[5, 35]}, \uline{[37, 38]}$ \\
$S_3:$ $\uline{[45, 60]}$

\vgap

\noindent The notation $\uline{[x, y]}$ represents a block holding integers $x$ and $y$. It is clear that $S_1$, $S_2$, and $S_3$ occupy 3, 2, and 1 block, respectively.

\vgap

Demonstrate how to merge $S_1$, $S_2$, and $S_3$ into one sorted file using the merging algorithm taught in the class. What is the I/O cost of the algorithm?

\begin{sol}
\extraspacing {\bf Solution.} The merging algorithm allocates (i) one memory block as the input buffer to read $S_i$, for each $i \in [1, 3]$, and (ii) one memory block as the output buffer. In the beginning, the algorithm loads the first block of each input file into memory, whose content is shown below:

\begin{center}
    Memory: $\uline{[1, 10]}, \uline{[5, 35]}, \uline{[45, 60]}, \uline{[\hspace{10mm}]}$
\end{center}

\noindent As long as no input buffer is empty, the algorithm moves the smallest integer in the 3 input buffers to the output buffer. This yields:

\begin{center}
    Memory: $\uline{[\hspace{5mm}, 10]}, \uline{[\hspace{5mm}, 35]}, \uline{[45, 60]}, \uline{[1, 5]}$
\end{center}

\noindent Now the output buffer is full and thus flushed to the disk:

\begin{center}
    Memory: $\uline{[\hspace{5mm}, 10]}, \uline{[\hspace{5mm}, 35]}, \uline{[45, 60]}, \uline{[\hspace{10mm}]}$ \\
    Output file on disk: $\uline{[1, 5]}$ \\
\end{center}

\noindent Moving the next smallest element in the input buffers to the output buffer yields:

\begin{center}
    Memory: $\uline{[\hspace{10mm}]}, \uline{[\hspace{5mm}, 35]}, \uline{[45, 60]}, \uline{[10, \hspace{5mm}]}$
\end{center}

\noindent The first input buffer is full, prompting the algorithm to read the next page of $S_1$:

\begin{center}
    Memory: $\uline{[20, 30]}, \uline{[\hspace{5mm}, 35]}, \uline{[45, 60]}, \uline{[10, \hspace{5mm}]}$
\end{center}

\noindent The next few steps of the algorithm are straightforward:

\begin{center}
    Memory: $\uline{[\hspace{5mm}, 30]}, \uline{[\hspace{5mm}, 35]}, \uline{[45, 60]}, \uline{[10, 20]}$ \\
    $\Rightarrow$ \\
    Memory: $\uline{[\hspace{5mm}, 30]}, \uline{[\hspace{5mm}, 35]}, \uline{[45, 60]}, \uline{[\hspace{10mm}]}$ \\
    Output file on disk: $\uline{[1, 5]}, \uline{[10, 20]}$ \\
    $\Rightarrow$ \\
    Memory: $\uline{[\hspace{10mm}]}, \uline{[\hspace{5mm}, 35]}, \uline{[45, 60]}, \uline{[30, \hspace{5mm}]}$
\end{center}

\noindent Reading the next page of $S_1$ gives:

\begin{center}
    Memory: $\uline{[40, 50]}, \uline{[\hspace{10mm}]}, \uline{[45, 60]}, \uline{[30, 35]}$ \\
    $\Rightarrow$ \\
    Memory: $\uline{[40, 50]}, \uline{[\hspace{10mm}]}, \uline{[45, 60]}, \uline{[\hspace{10mm}]}$ \\
    Output file on disk: $\uline{[1, 5]}, \uline{[10, 20]}, \uline{[30, 35]}$ \\
\end{center}

\noindent Now we reading the next page of $S_2$:

\begin{center}
    Memory: $\uline{[40, 50]}, \uline{[37, 38]}, \uline{[45, 60]}, \uline{[\hspace{10mm}]}$ \\
    $\Rightarrow$ \\
    Memory: $\uline{[40, 50]}, \uline{[\hspace{10mm}]}, \uline{[45, 60]}, \uline{[37, 38]}$ \\
    $\Rightarrow$ \\
    Memory: $\uline{[40, 50]}, \uline{[\hspace{10mm}]}, \uline{[45, 60]}, \uline{[\hspace{10mm}]}$ \\
    Output file on disk: $\uline{[1, 5]}, \uline{[10, 20]}, \uline{[30, 35]}, \uline{[37, 38]}$ \\
\end{center}

\noindent No need to replenish the input buffer of $S_2$ because this file has been exhausted. The remaining execution should be straightforward:

\begin{center}
    Memory: $\uline{[\hspace{5mm}, 50]}, \uline{[\hspace{10mm}]}, \uline{[\hspace{5mm}, 60]}, \uline{[40, 45]}$ \\
    $\Rightarrow$ \\
    Memory: $\uline{[\hspace{5mm}, 50]}, \uline{[\hspace{10mm}]}, \uline{[\hspace{5mm}, 60]}, \uline{[\hspace{10mm}]}$ \\
    Output file on disk: $\uline{[1, 5]}, \uline{[10, 20]}, \uline{[30, 35]}, \uline{[37, 38]}, \uline{[40, 45]}$ \\
    $\Rightarrow$ \\
    Memory: $\uline{[\hspace{10mm}]}, \uline{[\hspace{10mm}]}, \uline{[\hspace{10mm}]}, \uline{[50, 60]}$ \\
    $\Rightarrow$ \\
    Memory: $\uline{[\hspace{10mm}]}, \uline{[\hspace{10mm}]}, \uline{[\hspace{10mm}]}, \uline{[\hspace{10mm}]}$ \\
    Output file on disk: $\uline{[1, 5]}, \uline{[10, 20]}, \uline{[30, 35]}, \uline{[37, 38]}, \uline{[40, 45]}, \uline{[50, 60]}$
\end{center}

The total I/O cost is 12 because every block in the input files is read once and every block in the output file is written once.

\end{sol}


\extraspacing {\bf Problem 2.} Assume that each disk block can hold 2 integers, and the memory has 3 blocks. You are given a set $S$ stored on the disk in 9 blocks as shown below:

\vgap

\noindent $S: \uline{[50, 30]}, \uline{[80, 20]}, \uline{[10, 70]}, \uline{[40, 60]}, \uline{[55, 25]}, \uline{[90, 5]}, \uline{[85, 95]}, \uline{[35, 15]}, \uline{[65, 75]}$

\vgap

\noindent Suppose that we execute the initial step of the external sort algorithm taught in the class. Show all the sorted runs produced by this step. What is the I/O cost of this step?

\begin{sol}

\extraspacing {\bf Solution.} The initial step reads 3 blocks of $S$ into memory at a time, sort them (in memory), and write the sorted list into a sorted run. Therefore, the first sorted run is:

\begin{center}
    $\uline{[10, 20]}, \uline{[30, 50]}, \uline{[70, 80]}$,
\end{center}

\noindent the second sorted run is

\begin{center}
    $\uline{[5, 25]}, \uline{[40, 55]}, \uline{[60, 90]}$,
\end{center}

\noindent and the last sorted run is

\begin{center}
    $\uline{[15, 35]}, \uline{[60, 75]}, \uline{[85, 95]}$.
\end{center}

The I/O cost is 18 (9 read I/Os + 9 write I/Os).

\end{sol}


\extraspacing {\bf Problem 3.} Continuing on Problem 2, now execute the first merging step on the sorted runs you obtained. Show the sorted runs at the end of this step. What is the I/O cost of this step?

\begin{sol}
\extraspacing {\bf Solution.} The merging step combines two sorted runs from the previous step into a single new sorted run at a time. Hence, the first new sorted run is:

\begin{center}
    $\uline{[5, 10]}, \uline{[20, 25]}, \uline{[30, 40]}, \uline{[50, 55]}, \uline{[60, 70]}, \uline{[80, 90]}$.
\end{center}

\noindent The 3rd sorted from the previous step has no other sorted run to be combined with. Hence, it is taken as the second new sorted run directly:

\begin{center}
    $\uline{[15, 35]}, \uline{[60, 75]}, \uline{[85, 95]}$.
\end{center}

The total number of I/Os is 12 (6 read I/Os + 6 write I/Os).

\end{sol}


\extraspacing {\bf Problem 4.} Let $S$ be a set of integers stored in 10000 blocks. If the memory has $M = 10$ blocks. How many merging steps are required to sort $S$?

\begin{sol}
\extraspacing {\bf Solution.} The initial step creates $10000/10 = 1000$ sorted runs. Each merging step combines $M - 1 = 99$ sorted runs into a single new sorted run. Therefore, there are $\lc 1000/99 \rc = 11$ sorted runs after the first merging step. Another merging step will complete the sort. The answer is therefore 2.
\end{sol}

\extraspacing {\bf Problem 5.} Let $S$ be a set of integers stored in $10^9$ blocks. If we sort $S$ using the external sort algorithm, what is the smallest number $M$ of memory blocks required for the algorithm to incur only one merging step?

\begin{sol}
\extraspacing {\bf Solution.} The initial step creates $\lc 10^9 / M \rc$ sorted runs. If sorting needs to be completed with only one merging step, we need
\myeqn{
    \lc 10^9 / M \rc \le M - 1. \nn
}
Solving the inequality for integer $M$ yields $M \ge 31624$.

\end{sol}

\begin{center}
    \uline{Critical Thinking}
\end{center}

\extraspacing {\bf Problem 6} Let $R(A, B)$ be a relation with attributes $A$ and $B$. No two tuples in $R$ are identical. Explain how to sort $R$ according to the following order: rank tuple $t_1$ before tuple $t_2$ if (i) $t_1.A < t_2.A$ or (ii) $t_1.A = t_2.A$ and $t_1.B < t_2.B$.

\extraspacing {\bf Problem 7} Let $S$ be a {\em bag} (a.k.a.\ multi-set) of integers, namely, $S$ may contain duplicate integers. Adapt the external sort algorithm to sort $S$.

\extraspacing {\bf Problem 8} Let $S$ be a {\em bag} (a.k.a.\ multi-set) of integers, namely, $S$ may contain duplicate integers. Adapt the external sort algorithm to output a disk file containing the {\em distinct} integers of $S$ in ascending order.

\extraspacing {\bf Problem 9} Discuss how to process the following query on a relation $R(A, B)$:
\mytab{
    \> \ttt{SELECT $A$, MAX($B$) FROM $R$ GROUP BY $A$} \\
}


\end{document}
