\documentclass{beamer}

\usetheme{Warsaw}
%\usetheme{CambridgeUS}

% modification history
% created on 18 sep 2011
% modified on 25 mar

%\usepackage{amsfonts, amsmath, amssymb}

%\setbeamertemplate{theorems}[numbered]
%\setbeamertemplate{theorems}[ams style] 
\usepackage[skins,breakable]{tcolorbox}
%\usepackage[normalem]{ulem}

%\usefonttheme[onlymath]{serif}                     // change the style of math font 

%=============set slide number=================
\addtobeamertemplate{navigation symbols}{}{
    \usebeamerfont{footline}
    \usebeamercolor[fg]{footline}
    \hspace{1em}
    \insertframenumber/\inserttotalframenumber
}
\setbeamercolor{footline}{fg=black}
\setbeamerfont{footline}{series=\bfseries}


%=============set footline=====================
\setbeamertemplate{footline}
{
  \leavevmode%
  \hbox{%
  \begin{beamercolorbox}[wd=.55\paperwidth,ht=2.25ex,dp=1ex,center]{author in head/foot}%
    \usebeamerfont{author in head/foot}\insertshortauthor
  \end{beamercolorbox}%
  \begin{beamercolorbox}[wd=.45\paperwidth,ht=2.25ex,dp=1ex,center]{title in head/foot}%
    \usebeamerfont{title in head/foot}\insertshorttitle
  \end{beamercolorbox}}%
  \vskip0pt%
}

%creating a rectangle box def
\newtcbox{\mybox}[1][red]{arc=0pt,outer arc=0pt,colback=#1!10!white,colframe=#1!50!black, boxsep=0pt,left=1pt,right=1pt,top=2pt,bottom=2pt,boxrule=0pt,bottomrule=1pt,toprule=1pt}

\newtcbox{\xmybox}[1][red]{arc=7pt,colback=#1!10!white,colframe=#1!50!black,before upper={\rule[-3pt]{0pt}{10pt}},boxrule=1pt,boxsep=0pt,left=6pt,right=6pt,top=2pt,bottom=2pt}
%the ``on line'' option doesn't work. so omitting it

%===== spacing =====

\def\extraspacing{\vspace{2mm} \noindent}
\def\vgap{\vspace{2mm}}
\def\hgap{\textrm{\hspace{1mm}}}

%===== tabbing =====

\def\tab{\hspace{2mm}}
\def\tabpos{\hspace{4mm} \= \hspace{4mm} \= \hspace{4mm} \= \hspace{4mm} \=
\hspace{4mm} \= \hspace{4mm} \= \hspace{4mm} \= \hspace{4mm} \= \hspace{4mm}
\kill}
\newcommand{\mytab}[1]{\begin{tabbing}\tabpos #1\end{tabbing}}

%===== blocks =====

% \newtheorem{theorem}{Theorem}
% \newtheorem{lemma}{Lemma}
% \newtheorem{corollary}{Corollary}
% \newtheorem{proposition}{Proposition}
% \newtheorem{definition}{Definition}
% \newtheorem{problem}{Problem}

\newcommand{\cbox}[2]{\begin{tcolorbox}[arc=0mm, colframe=#1!50!black, colback=#1!10!white]#2\end{tcolorbox}}
\newcommand{\minipg}[2]{\begin{center}\begin{minipage}{#1}#2\end{minipage}\end{center}}
\newcommand{\myfrm}[1]{\begin{frame}\begin{small}#1\end{small}\end{frame}} 
\newcommand{\myitems}[1]{\begin{itemize}#1\end{itemize}}
\newcommand{\myenums}[1]{\begin{enumerate}#1\end{enumerate}}
\newcommand{\myfig}[1]{\begin{figure}\centering #1\end{figure}}
    
%===== math macros =====
\newcommand{\bm}[1]{\textrm{\boldmath${#1}$}}
%\newcommand{\smat}[2]{\left[\begin{tabular}{#1}#2\end{tabular}\right]}
%\newcommand{\bmat}[2]{\left|\begin{tabular}{#1}#2\end{tabular}\right|}
\newcommand{\bmat}[1]{\begin{bmatrix}#1\end{bmatrix}}
\newcommand{\vmat}[1]{\begin{vmatrix}#1\end{vmatrix}}
\newcommand{\myeqn}[1]{\begin{eqnarray}#1\end{eqnarray}}
\newcommand{\set}[1]{\{#1\}}

\def\eps{\epsilon}
\def\fr{\frac}
\def\lc{\lceil}
\def\lf{\lfloor}
\def\rc{\rceil}
\def\rf{\rfloor}
\def\Pr{\textrm{\boldmath$Pr$}}
\def\expt{\textrm{\boldmath$E$}}
\def\real{\mathbb{R}}
\def\int{\mathbb{Z}}
\def\*{\star}
\def\tO{\tilde{O}}

\DeclareMathOperator*{\argmin}{arg\,min}
\DeclareMathOperator*{\polylg}{polylg}
\DeclareMathOperator*{\polylog}{polylog}
\DeclareMathOperator*{\intr}{\cap}

\def\nn{\nonumber}
\def\mit{\mathit}


%===== misc =====

\def\done{\hspace*{\fill} $\framebox[2mm]{}$}	% end of proof
\def\ttt{\texttt}

%===== coloring =====
\newcommand{\red}[1]{\textcolor{red}{#1}}
\newcommand{\bred}[1]{\textcolor{red}{\bf #1}}
\newcommand{\blue}[1]{\textcolor{blue}{\bf #1}}

\usepackage{color}
\usepackage{graphicx}
\usepackage{multirow}
\usepackage{wrapfig}
\usepackage[skins,breakable]{tcolorbox}

\def\done{\hfill$\square$}
\def\ttt{\texttt}
\def\vgap{\vspace{5mm}}

\def\abt{\ttt{ABORT}}
\def\best{\mit{best}}
\def\cmt{\ttt{COMMIT}}
\def\rd{\ttt{READ}}
\def\size{\mit{size}}
\def\sort{\mit{sort}}
\def\wt{\ttt{WRITE}}

\title[DATABASE SYSTEM PRINCIPLES]{Transactions 1:\\ Fundamentals}

\author[Yufei Tao @ NTU]{Yufei Tao}
\institute[]{\url{https://www.cse.cuhk.edu.hk/~taoyf}}
\date{}

% \def\dtm{\mathit{d\mbox{-}tm}}
% \def\ftm{\mathit{f\mbox{-}tm}}
\def\bestext{\mathit{best\mbox{-}ext}}

\begin{document}
%-------------------------------------------------------------
\begin{frame}
    \titlepage
%     \begin{tcolorbox}[arc=0mm, colframe=green!50!black, colback=green!10!white] 
%     \end{tcolorbox}
\end{frame}
%-------------------------------------------------------------
\begin{frame}
\begin{small}
    Today, we will start our discussion on \bred{transactions}, focusing on how database systems execute multiple data-access ``programs'' concurrently without violating data consistency.
    %\vgap
\end{small}    
\end{frame}
%-------------------------------------------------------------
\myfrm{
    \cbox{blue}{
        We define the \blue{state} of a database as the content of all the tables in the database.
    }

    \vgap

    \cbox{blue}{
        A database state is \blue{consistent} if the state satisfies all the constraints imposed on the tables.
    }
    \bred{Constraint examples}: primary key constraints, foreign key constraints, ``not NULL'', user-defined constraints, and so on.
}
%-------------------------------------------------------------
\myfrm{
    \cbox{blue}{
        A \blue{transaction} is a sequence of instructions submitted to a database with the requirement that \bred{either} the entire sequence is executed \bred{or} nothing from the sequence should be executed.
    }
    This is the \blue{all-or-nothing} requirement.

    \vgap

    \cbox{green}{
        \blue{Example:}

        \begin{center}
        \begin{tabular}{c|c}
            $\red{T}$ & remark\\
            \hline
            \rd($A$) & read a tuple's value into a memory copy \\
            $A = A + 1$ &\\
            \wt($A$) & update the tuple \\
            \cmt & inform the system ``ready to finish''
        \end{tabular}
        \end{center}

        %\vgap

        If the transaction is executed, the tuple's value must increase by 1.
    }
}
%-------------------------------------------------------------
\myfrm{
    \cbox{blue}{
        A transaction is \blue{committed} if it is executed in full or \blue{aborted} if it is terminated during the execution. An abort may be caused by the transaction itself or by the system.
    }
    \cbox{green}{
        \blue{Example:}

        \begin{center}
        \begin{tabular}{c|c}
            $\red{T}$ & remark\\
            \hline
            \rd($A$) & \\
            $A = A + 1$ &\\
            \wt($A$) & \bred{suppose that a system error occurs here}
        \end{tabular}
        \end{center}

        %\vgap

        The transaction must be aborted. The tuple's value must be restored to its original value (before the transaction).
    }

    \cbox{blue}{
        A \blue{rollback} process cancels all the modifications to the database performed by the transaction.
    }
}
%-------------------------------------------------------------
\myfrm{
    \cbox{blue}{
        A transaction is \blue{legal} if it always leaves a \bred{consistent} database state upon completion.
    }
    All transactions in our discussion are assumed to be legal.

    \vgap

    \cbox{red}{
        \bred{Note:} The database state is allowed to be inconsistent \bred{during} a transaction.
    }
}
%-------------------------------------------------------------
\myfrm{
    \cbox{blue}{
        A \blue{schedule} is a sequence of instructions executed by the database to run one or more transactions.
    }

    \cbox{green}{
        \blue{Example:}
        \begin{center}
        \begin{tabular}{c|c}
            $\red{T_1}$ & $\red{T_2}$\\
            \hline
            \rd($A$) & \\
            & \rd($B$) \\
            $A = A + 1$ &\\
            \wt($A$) & \\
            & $B = B + 1$ \\
            & \wt($B$) \\
            \cmt & \\
            & \cmt
        \end{tabular}
        \end{center}
    }
}
%-------------------------------------------------------------
\myfrm{
    \cbox{blue}{
        A schedule is \blue{serial} if there is no interleaving among the instructions of different transactions. Otherwise, it is a \blue{concurrent} schedule.
    }

    \cbox{green}{
        \blue{Example:} A serial schedule
        \begin{center}
        \begin{tabular}{c|c}
            $\red{T_1}$ & $\red{T_2}$\\
            \hline
            \rd($A$) & \\
            $A = A + 1$ &\\
            \wt($A$) & \\
            \cmt & \\
            & \rd($B$) \\
            & $B = B + 1$ \\
            & \wt($B$) \\
            & \cmt
        \end{tabular}
        \end{center}
    }

    \cbox{blue}{
        Serial schedules always leave a consistent database state.
    }
}
%-------------------------------------------------------------
\myfrm{
    \cbox{blue}{
        A schedule is \blue{recoverable} if it guarantees the following: if transaction $\red{T}$ reads a value written by another transaction $\red{T'}$, then \bred{$T$ commits after $T'$}.
    }
    \cbox{green}{
        \blue{Example:} The schedule below is \bred{not} recoverable.
        \begin{center}
        \begin{tabular}{c|c}
            $\red{T_1}$ & $\red{T_2}$\\
            \hline
            \rd($A$) & \\
            $A = A + 1$ &\\
            \wt($A$) & \\
            & \rd($A$) \\
            & $A = A + 1$ \\
            & \wt($A$) \\
            & \cmt \\
            \cmt &
        \end{tabular}
        \end{center}
    }

    \cbox{red}{
        All schedules in DB systems need to be recoverable! (\blue{Think:} why?)
    }
}
%-------------------------------------------------------------
\myfrm{
    \cbox{blue}{
        Two schedules are \blue{equivalent} if they always result in the same database state.
    }

    \cbox{green}{
        \blue{Example:}
        \begin{center}
        \begin{tabular}{c|c}
            $\red{T_1}$ & $\red{T_2}$\\
            \hline
            & \rd($A$) \\
            & $A = A + 1$ \\
            & \wt($A$) \\
            \rd($A$) & \\
            $B = A$ & \\
            \wt($B$) &\\
            & \cmt \\
            \cmt &
        \end{tabular}
        \end{center}

        The schedule is equivalent to the serial schedule $(T_2, T_1)$, but \bred{not} equivalent to the serial schedule $(T_1, T_2)$.
    }
}
%-------------------------------------------------------------
\myfrm{
    \cbox{blue}{
        A schedule is \blue{serializable} if it is equivalent to \bred{some} serial schedule. Otherwise, it is \blue{non-serializable}.
    }

    \cbox{green}{
        \blue{Example:} The following schedule is serializable.
        \begin{center}
        \begin{tabular}{c|c}
            $\red{T_1}$ & $\red{T_2}$\\
            \hline
            & \rd($A$) \\
            & $A = A + 1$ \\
            & \wt($A$) \\
            \rd($A$) & \\
            $B = A$ & \\
            \wt($B$) &\\
            & \cmt \\
            \cmt &
        \end{tabular}
        \end{center}
    }
}
%-------------------------------------------------------------
\myfrm{
    \cbox{green}{
        \blue{Example:} The following schedule is non-serialzable.

        \begin{center}
        \begin{tabular}{c|c}
            $\red{T_1}$ & $\red{T_2}$\\
            \hline
            \rd($A$) & \\
            $A = A + 1$ &\\
            & \rd($A$) \\
            & $A = A + 1$ \\
            \wt($A$) & \\
            & \wt($A$) \\
            & \cmt \\
            \cmt &
        \end{tabular}
        \end{center}

        \vgap

        If $A$ was 0 before running the schedule, what is its value afterwards?
    }
}
%-------------------------------------------------------------

\end{document} 



