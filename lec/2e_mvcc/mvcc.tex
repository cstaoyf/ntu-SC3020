\documentclass{beamer}

\usetheme{Warsaw}
%\usetheme{CambridgeUS}

% modification history
% created on 18 sep 2011
% modified on 25 mar

%\usepackage{amsfonts, amsmath, amssymb}

%\setbeamertemplate{theorems}[numbered]
%\setbeamertemplate{theorems}[ams style] 
\usepackage[skins,breakable]{tcolorbox}
%\usepackage[normalem]{ulem}

%\usefonttheme[onlymath]{serif}                     // change the style of math font 

%=============set slide number=================
\addtobeamertemplate{navigation symbols}{}{
    \usebeamerfont{footline}
    \usebeamercolor[fg]{footline}
    \hspace{1em}
    \insertframenumber/\inserttotalframenumber
}
\setbeamercolor{footline}{fg=black}
\setbeamerfont{footline}{series=\bfseries}


%=============set footline=====================
\setbeamertemplate{footline}
{
  \leavevmode%
  \hbox{%
  \begin{beamercolorbox}[wd=.55\paperwidth,ht=2.25ex,dp=1ex,center]{author in head/foot}%
    \usebeamerfont{author in head/foot}\insertshortauthor
  \end{beamercolorbox}%
  \begin{beamercolorbox}[wd=.45\paperwidth,ht=2.25ex,dp=1ex,center]{title in head/foot}%
    \usebeamerfont{title in head/foot}\insertshorttitle
  \end{beamercolorbox}}%
  \vskip0pt%
}

%creating a rectangle box def
\newtcbox{\mybox}[1][red]{arc=0pt,outer arc=0pt,colback=#1!10!white,colframe=#1!50!black, boxsep=0pt,left=1pt,right=1pt,top=2pt,bottom=2pt,boxrule=0pt,bottomrule=1pt,toprule=1pt}

\newtcbox{\xmybox}[1][red]{arc=7pt,colback=#1!10!white,colframe=#1!50!black,before upper={\rule[-3pt]{0pt}{10pt}},boxrule=1pt,boxsep=0pt,left=6pt,right=6pt,top=2pt,bottom=2pt}
%the ``on line'' option doesn't work. so omitting it

%===== spacing =====

\def\extraspacing{\vspace{2mm} \noindent}
\def\vgap{\vspace{2mm}}
\def\hgap{\textrm{\hspace{1mm}}}

%===== tabbing =====

\def\tab{\hspace{2mm}}
\def\tabpos{\hspace{4mm} \= \hspace{4mm} \= \hspace{4mm} \= \hspace{4mm} \=
\hspace{4mm} \= \hspace{4mm} \= \hspace{4mm} \= \hspace{4mm} \= \hspace{4mm}
\kill}
\newcommand{\mytab}[1]{\begin{tabbing}\tabpos #1\end{tabbing}}

%===== blocks =====

% \newtheorem{theorem}{Theorem}
% \newtheorem{lemma}{Lemma}
% \newtheorem{corollary}{Corollary}
% \newtheorem{proposition}{Proposition}
% \newtheorem{definition}{Definition}
% \newtheorem{problem}{Problem}

\newcommand{\cbox}[2]{\begin{tcolorbox}[arc=0mm, colframe=#1!50!black, colback=#1!10!white]#2\end{tcolorbox}}
\newcommand{\minipg}[2]{\begin{center}\begin{minipage}{#1}#2\end{minipage}\end{center}}
\newcommand{\myfrm}[1]{\begin{frame}\begin{small}#1\end{small}\end{frame}} 
\newcommand{\myitems}[1]{\begin{itemize}#1\end{itemize}}
\newcommand{\myenums}[1]{\begin{enumerate}#1\end{enumerate}}
\newcommand{\myfig}[1]{\begin{figure}\centering #1\end{figure}}
    
%===== math macros =====
\newcommand{\bm}[1]{\textrm{\boldmath${#1}$}}
%\newcommand{\smat}[2]{\left[\begin{tabular}{#1}#2\end{tabular}\right]}
%\newcommand{\bmat}[2]{\left|\begin{tabular}{#1}#2\end{tabular}\right|}
\newcommand{\bmat}[1]{\begin{bmatrix}#1\end{bmatrix}}
\newcommand{\vmat}[1]{\begin{vmatrix}#1\end{vmatrix}}
\newcommand{\myeqn}[1]{\begin{eqnarray}#1\end{eqnarray}}
\newcommand{\set}[1]{\{#1\}}

\def\eps{\epsilon}
\def\fr{\frac}
\def\lc{\lceil}
\def\lf{\lfloor}
\def\rc{\rceil}
\def\rf{\rfloor}
\def\Pr{\textrm{\boldmath$Pr$}}
\def\expt{\textrm{\boldmath$E$}}
\def\real{\mathbb{R}}
\def\int{\mathbb{Z}}
\def\*{\star}
\def\tO{\tilde{O}}

\DeclareMathOperator*{\argmin}{arg\,min}
\DeclareMathOperator*{\polylg}{polylg}
\DeclareMathOperator*{\polylog}{polylog}
\DeclareMathOperator*{\intr}{\cap}

\def\nn{\nonumber}
\def\mit{\mathit}


%===== misc =====

\def\done{\hspace*{\fill} $\framebox[2mm]{}$}	% end of proof
\def\ttt{\texttt}

%===== coloring =====
\newcommand{\red}[1]{\textcolor{red}{#1}}
\newcommand{\bred}[1]{\textcolor{red}{\bf #1}}
\newcommand{\blue}[1]{\textcolor{blue}{\bf #1}}

\usepackage{color}
\usepackage{graphicx}
\usepackage{multirow}
\usepackage{wrapfig}
\usepackage[skins,breakable]{tcolorbox}

\def\done{\hfill$\square$}
\def\ttt{\texttt}
\def\vgap{\vspace{5mm}}

\def\abt{\ttt{ABORT}}
\def\best{\mit{best}}
\def\cmt{\ttt{COMMIT}}
\def\iddel{ID_\mit{del}}
\def\idins{ID_\mit{ins}}
\def\ins{\ttt{INSERT}}
\def\rd{\ttt{READ}}
\def\size{\mit{size}}
\def\slock{\ttt{S-LOCK}}
\def\sort{\mit{sort}}
\def\upg{\ttt{UPGRADE}}
\def\ulock{\ttt{U-LOCK}}
\def\unlock{\ttt{UNLOCK}}
\def\wt{\ttt{WRITE}}
\def\xlock{\ttt{X-LOCK}}

\title[DATABASE SYSTEM PRINCIPLES]{Transactions 5:\\ Multi-Version Concurrency Control}

\author[Yufei Tao @ NTU]{Yufei Tao}
\institute[]{\url{https://www.cse.cuhk.edu.hk/~taoyf}}
\date{}

% \def\dtm{\mathit{d\mbox{-}tm}}
% \def\ftm{\mathit{f\mbox{-}tm}}
\def\bestext{\mathit{best\mbox{-}ext}}

\begin{document}
%-------------------------------------------------------------
\begin{frame}
    \titlepage
%     \begin{tcolorbox}[arc=0mm, colframe=green!50!black, colback=green!10!white] 
%     \end{tcolorbox}
\end{frame}
%-------------------------------------------------------------
\begin{frame}
\begin{small}
    Today, we will discuss how to enforce the \bred{read committed} isolation level with \blue{multi-version concurrency control} (MVCC).

    \vgap

    \cbox{blue}{
        Recall that the serializable isolation level should be enforced for critical transactions. For non-critical transactions (e.g., browsing books in a library), the read committed isolation level may suffice. In fact, this is the default isolation level in PostgreSQL and ORACLE (both are well-known database systems).
    }
\end{small}    
\end{frame}
%-------------------------------------------------------------
\myfrm{
    \cbox{blue}{
        Each transaction is assigned a unique \blue{ID}, with the requirement that a transaction $\red{T_1}$ starts before another transaction $\red{T_2}$, the ID of $T_1$ is smaller than that of $T_2$.
    }
}
%-------------------------------------------------------------
\myfrm{
    \xmybox{Multi-Versioning}

    %\vgap

    \cbox{blue}{
        For each tuple $\red{t}$, the database stores a \blue{history}, which is a sequence of \blue{records} $\red{r_1}, \red{r_2}, ..., \red{r_\ell}$ for some $\red{\ell} \ge 1$ such that:
        \myitems{
            \item each $\red{r_i}$ ($1 \le \red{i} \le \ell$) has the form $(\red{v}, \red{\idins}, \red{\iddel})$, where
            \myitems{
                \item $v$ is version of $t$ that ever existed in history;
                \item $\idins$ is the ID of the transaction that created $v$;
                \item $\iddel$ is the ID of the transaction that deleted $v$ or \ttt{\red{NULL}} if $v$ has never been deleted;
            }

            \item $r_{i+1}.\idins = r_i.\iddel$ for every $\red{i} \in [1, \ell - 1]$.
        }
    }

    \cbox{green}{
        \blue{Invariant:} If $\ell \ge 2$, the transactions with IDs $r_1.\idins$, $r_2.\idins$, ..., $r_{\ell-1}.\idins$ must have committed.
    }
}
%-------------------------------------------------------------
\myfrm{
    \xmybox{Multi-Versioning}

    %\vgap

    \cbox{green}{
        \blue{Example:} Let $\red{t}$ be a tuple in the relation $R(X, Y)$ with the history:

        \begin{center}
             (($x_1, y_1$), 100, 103) \\
             (($x_2, y_2$), 103, 110) \\
             (($x_3, y_3$), 110, \ttt{NULL}).
        \end{center}

        This indicates:
        \myitems{
            \item $t$ was first inserted as $(x_1, y_1)$ by transaction (with ID) 100.
            \item $t$ was then modified to $(x_2, y_2)$ by transaction 103.
            \item $t$ was then modified again to $(x_3, y_3)$ by transaction 110 and still exists till now.
        }

        By the invariant, transactions 100 and 103 must have committed.
    }
}
%-------------------------------------------------------------
\myfrm{
    \xmybox{Concurrency Control}

    %\vgap

   %\vgap

    \cbox{blue}{
        \blue{Rule 1:} Reads are always granted (no locks required).
    }

    \cbox{blue}{
        \blue{Rule 2:} Let $\red{T}$ be a transaction with id $\red{ID_T}$ and $\red{t}$ be a tuple. Suppose that $T$ performs $\red{\rd(t)}$, and the \bred{current} history of $t$ is $\red{r_1}, \red{r_2}, ..., \red{r_\ell}$. The database acts as follows:
        \myitems{
            \item \blue{if} $\red{r_\ell.\idins = ID_T}$, then return $r_\ell.v$;
            \item \blue{elseif} \bred{$r_\ell.\iddel \ne \ttt{NULL}$}, then return nothing;
            \item \blue{elseif} \bred{the transaction with ID $r_\ell.\idins$ has committed}, then return $r_\ell.v$;
            \item \blue{elseif} \bred{$\ell \ge 2$}, then return $r_{\red{\ell-1}}.v$;
            \item \blue{else} return nothing.
        }
    }
}
%-------------------------------------------------------------
\myfrm{
    \xmybox{Concurrency Control}

    %\vgap

    \cbox{green}{
        \blue{Example:} Let $\red{t}$ be a tuple in the relation $R(X, Y)$ with the history:

        \begin{center}
             (($x_1, y_1$), 100, 103) \\
             (($x_2, y_2$), 103, 110) \\
             (($x_3, y_3$), 110, \ttt{NULL}).
        \end{center}

        \myitems{
            \item If transaction 110 issues $\rd(t)$, it gets $(x_3, y_3)$.
            \item If trans.\ 109 issues $\rd(t)$ with trans.\ 110 already committed, it gets $(x_3, y_3)$, i.e., the \bred{last committed version}.
            \item If trans.\ 109 issues $\rd(t)$ with trans.\ 110 still running, it gets $(x_2, y_2)$, i.e., \bred{last committed version}.
        }
    }
}
%-------------------------------------------------------------
\myfrm{
    \xmybox{Concurrency Control}

    %\vgap

   %\vgap

    \cbox{blue}{
        \blue{Rule 3:} Before a transaction can write to a tuple $\red{t}$, it must be holding an \bred{X-lock} on $t$. To obtain such a lock, a transaction needs to make a \blue{lock request}. The database grants the request if currently there are no locks on $t$. Otherwise, the request is denied, and the transaction is suspended until its request is granted.
    }

    \cbox{blue}{
        \blue{Rule 4:} Let $\red{T}$ be a transaction with id $\red{ID_T}$ that holds an X-lock on a tuple $\red{t}$. Suppose that $T$ performs $\red{\wt(t, v)}$. The database acts as follows:
        \myitems{
            \item If $t$ is new (i.e., no history), create a history of $t$ with only one record $\red{(v, ID_T, \ttt{NULL})}$.
            \item Otherwise, let $\red{r_1}, \red{r_2}, ..., \red{r_\ell}$ be the \bred{current} history of $t$. Set $\red{r_\ell.\iddel = ID_T}$ and append a record $\red{(v, ID_T, \ttt{NULL})}$ to the history.
        }
        %, and
    }
}
%-------------------------------------------------------------
\myfrm{
    \xmybox{Concurrency Control}

    %\vgap

   \vgap

    \cbox{blue}{
        \blue{Rule 5:} A transaction releases all locks after committing.
    }

}
%-------------------------------------------------------------
\myfrm{
    \cbox{green}{
        \blue{Example 1} Let $\red{t}$ be a tuple of relation $\red{R(A)}$ whose history has a single record $(10, 100, \ttt{NULL})$. Transaction 100 has committed. Transactions $\red{T_1}$, $\red{T_2}$, and $\red{T_2}$ have IDs 101, 102, and 103, respectively. Assume that the database attempts to execute the instructions of $T_1$ and $T_2$ in the order shown below.

        \begin{center}
        \begin{tabular}{c|c|c|c}
            &$\red{T_1}$ ID 101 & $\red{T_2}$ ID 102 & $\red{T_3}$ ID 103\\
            \hline
            1& \rd($t$) & & \\
            2& & $\xlock(t)$ & \\
            3& & \wt($t$, \red{20}) & \\
            4& & & $\xlock(t)$ \\
            5& & & \wt($t$, \red{30}) \\
            6& $\rd(t)$ & \\
            7& & \rd$(t)$ \\
            8& & \cmt \\
            9&&&\cmt \\
            10& $\rd(t)$ & \\
            11&\cmt&
        \end{tabular}
        \end{center}

        We will walk through the execution in the next few slides.
    }
}
%-------------------------------------------------------------
\myfrm{
    \cbox{green}{
        \blue{Example 1 (cont.)}
        \begin{center}
        \begin{tabular}{c|c|c|c}
            &$\red{T_1}$ ID 101 & $\red{T_2}$ ID 102 & $\red{T_3}$ ID 103\\
            \hline
            $\red{\bm{\rightarrow}}$ 1& \rd($t$) & & \\
            2& & $\xlock(t)$ & \\
            3& & \wt($t$, \red{20}) & \\
            4& & & $\xlock(t)$ \\
            5& & & \wt($t$, \red{30}) \\
            6& $\rd(t)$ & \\
            7& & \rd$(t)$ \\
            8& & \cmt \\
            9&&&\cmt \\
            10& $\rd(t)$ & \\
            11&\cmt&
        \end{tabular}
        \end{center}

        $T_1$ reads $t = 10$ (Rules 1, 2).
    }
}
%-------------------------------------------------------------
\myfrm{
    \cbox{green}{
        \blue{Example 1 (cont.)}
        \begin{center}
        \begin{tabular}{c|c|c|c}
            &$\red{T_1}$ ID 101 & $\red{T_2}$ ID 102 & $\red{T_3}$ ID 103\\
            \hline
            $\red{\bm{\rightarrow}}$ 2& & $\xlock(t)$ & \\
            3& & \wt($t$, \red{20}) & \\
            4& & & $\xlock(t)$ \\
            5& & & \wt($t$, \red{30}) \\
            6& $\rd(t)$ & \\
            7& & \rd$(t)$ \\
            8& & \cmt \\
            9&&&\cmt \\
            10& $\rd(t)$ & \\
            11&\cmt&
        \end{tabular}
        \end{center}

        Granted (Rule 3).
    }
}
%-------------------------------------------------------------
\myfrm{
    \cbox{green}{
        \blue{Example 2 (cont.)}

\begin{center}
\begin{tabular}{c|c|c|c}
    &$\red{T_1}$ ID 101 & $\red{T_2}$ ID 102 & $\red{T_3}$ ID 103\\
    \hline
    $\red{\bm{\rightarrow}}$ 3& & \wt($t$, \red{20}) & \\
    4& & & $\xlock(t)$ \\
    5& & & \wt($t$, \red{30}) \\
    6& $\rd(t)$ & \\
    7& & \rd$(t)$ \\
    8& & \cmt \\
    9&&&\cmt \\
    10& $\rd(t)$ & \\
    11&\cmt&
\end{tabular}
\end{center}

        By Rule 4, the history of $t$ now has two records: \\
        $(10, 100, 102)$, $(20,102, \ttt{NULL})$.
    }
}
%-------------------------------------------------------------
\myfrm{
    \cbox{green}{
        \blue{Example 1 (cont.)}

        \begin{center}
\begin{tabular}{c|c|c|c}
    &$\red{T_1}$ ID 101 & $\red{T_2}$ ID 102 & $\red{T_3}$ ID 103\\
    \hline
    $\red{\bm{\rightarrow}}$ 4& & & $\xlock(t)$ \\
    5& & & \wt($t$, \red{30}) \\
    6& $\rd(t)$ & \\
    7& & \rd$(t)$ \\
    8& & \cmt \\
    9&&&\cmt \\
    10& $\rd(t)$ & \\
    11&\cmt&
\end{tabular}
\end{center}

        Denied (Rule 3) and $T_3$ is suspended.
    }
}
%-------------------------------------------------------------
\myfrm{
    \cbox{green}{
        \blue{Example 1 (cont.)}

        \begin{center}
\begin{tabular}{c|c|c|c}
    &$\red{T_1}$ ID 101 & $\red{T_2}$ ID 102 & $\red{T_3}$ ID 103\\
    \hline
    4& & & $\xlock(t)$ \\
    5& & & \wt($t$, \red{30}) \\
    $\red{\bm{\rightarrow}}$ 6& $\rd(t)$ & \\
    7& & \rd$(t)$ \\
    8& & \cmt \\
    9&&&\cmt \\
    10& $\rd(t)$ & \\
    11&\cmt&
\end{tabular}
\end{center}


        History of $t$:
        $(10, 100, 102)$, $(20,102, \ttt{NULL})$. \\

        $T_1$ reads $t = 10$ (Rules 1, 2), noticing that the trans.\ with ID 102 (i.e., $T_2$) is still running.
    }
}
%-------------------------------------------------------------
\myfrm{
    \cbox{green}{
        \blue{Example 1 (cont.)}
         \begin{center}
\begin{tabular}{c|c|c|c}
    &$\red{T_1}$ ID 101 & $\red{T_2}$ ID 102 & $\red{T_3}$ ID 103\\
    \hline
    4& & & $\xlock(t)$ \\
    5& & & \wt($t$, \red{30}) \\
$\red{\bm{\rightarrow}}$    7& & \rd$(t)$ \\
    8& & \cmt \\
    9&&&\cmt \\
    10& $\rd(t)$ & \\
    11&\cmt&
\end{tabular}
\end{center}

        History of $t$:
        $(10, 100, 102)$, $(20,102, \ttt{NULL})$. \\

        $T_2$ reads $t = 20$ (Rules 1, 2).
    }
}
%-------------------------------------------------------------
\myfrm{
    \cbox{green}{
        \blue{Example 1 (cont.)}
         \begin{center}
\begin{tabular}{c|c|c|c}
    &$\red{T_1}$ ID 101 & $\red{T_2}$ ID 102 & $\red{T_3}$ ID 103\\
    \hline
    4& & & $\xlock(t)$ \\
    5& & & \wt($t$, \red{30}) \\
$\red{\bm{\rightarrow}}$    8& & \cmt \\
    9&&&\cmt \\
    10& $\rd(t)$ & \\
    11&\cmt&
\end{tabular}
\end{center}

        History of $t$:
        $(10, 100, 102)$, $(20,102, \ttt{NULL})$. \\

        $T_2$ commits and releases the X-lock on $t$ (Rule 5).
    }
}
%-------------------------------------------------------------
\myfrm{
    \cbox{green}{
        \blue{Example 1 (cont.)}
         \begin{center}
\begin{tabular}{c|c|c|c}
    &$\red{T_1}$ ID 101 & $\red{T_2}$ ID 102 & $\red{T_3}$ ID 103\\
    \hline
$\red{\bm{\rightarrow}}$    4& & & $\xlock(t)$ \\
    5& & & \wt($t$, \red{30}) \\
    9&&&\cmt \\
    10& $\rd(t)$ & \\
    11&\cmt&
\end{tabular}
\end{center}

        History of $t$:
        $(10, 100, 102)$, $(20,102, \ttt{NULL})$. \\

        $T_3$ now resumes and holds the X-lock on $t$ (Rule 3).
    }
}
%-------------------------------------------------------------
\myfrm{
    \cbox{green}{
        \blue{Example 1 (cont.)}
         \begin{center}
\begin{tabular}{c|c|c|c}
    &$\red{T_1}$ ID 101 & $\red{T_2}$ ID 102 & $\red{T_3}$ ID 103\\
    \hline
$\red{\bm{\rightarrow}}$
    5& & & \wt($t$, \red{30}) \\
    9&&&\cmt \\
    10& $\rd(t)$ & \\
    11&\cmt&
\end{tabular}
\end{center}

        By Rule 4, the history of $t$ becomes \\
        $(10, 100, 102)$, $(20,102, 103)$, $(30,103, \ttt{NULL})$.
    }
}
%-------------------------------------------------------------
\myfrm{
    \cbox{green}{
        \blue{Example 1 (cont.)}
         \begin{center}
\begin{tabular}{c|c|c|c}
    &$\red{T_1}$ ID 101 & $\red{T_2}$ ID 102 & $\red{T_3}$ ID 103\\
    \hline
$\red{\bm{\rightarrow}}$
    9&&&\cmt \\
    10& $\rd(t)$ & \\
    11&\cmt&
\end{tabular}
\end{center}

        History of $t$:
        $(10, 100, 102)$, $(20,102, 103)$, $(30,103, \ttt{NULL})$. \\

                $T_3$ commits and releases the X-lock on $t$ (Rule 5).

    }
}
%-------------------------------------------------------------
\myfrm{
    \cbox{green}{
        \blue{Example 1 (cont.)}
         \begin{center}
\begin{tabular}{c|c|c|c}
    &$\red{T_1}$ ID 101 & $\red{T_2}$ ID 102 & $\red{T_3}$ ID 103\\
    \hline
$\red{\bm{\rightarrow}}$
    10& $\rd(t)$ & \\
    11&\cmt&
\end{tabular}
\end{center}

        History of $t$:
        $(10, 100, 102)$, $(20,102, 103)$, $(30,103, \ttt{NULL})$. \\

        $T_1$ reads $t = 30$ (Rules 1, 2).


    }
}
%-------------------------------------------------------------
\myfrm{
     \cbox{blue}{
        \blue{Theorem:} The schedule produced by the protocol we described is  recoverable and enforces the read committed isolation levels.
    }

    \vgap

     The theorem holds regardless of how the instructions of different transactions are interleaved.

     \vgap
}
%-------------------------------------------------------------
\end{document} 



