\documentclass{beamer}

\usetheme{Warsaw}
%\usetheme{CambridgeUS}

% modification history
% created on 18 sep 2011
% modified on 25 mar

%\usepackage{amsfonts, amsmath, amssymb}

%\setbeamertemplate{theorems}[numbered]
%\setbeamertemplate{theorems}[ams style] 
\usepackage[skins,breakable]{tcolorbox}
%\usepackage[normalem]{ulem}

%\usefonttheme[onlymath]{serif}                     // change the style of math font 

%=============set slide number=================
\addtobeamertemplate{navigation symbols}{}{
    \usebeamerfont{footline}
    \usebeamercolor[fg]{footline}
    \hspace{1em}
    \insertframenumber/\inserttotalframenumber
}
\setbeamercolor{footline}{fg=black}
\setbeamerfont{footline}{series=\bfseries}


%=============set footline=====================
\setbeamertemplate{footline}
{
  \leavevmode%
  \hbox{%
  \begin{beamercolorbox}[wd=.55\paperwidth,ht=2.25ex,dp=1ex,center]{author in head/foot}%
    \usebeamerfont{author in head/foot}\insertshortauthor
  \end{beamercolorbox}%
  \begin{beamercolorbox}[wd=.45\paperwidth,ht=2.25ex,dp=1ex,center]{title in head/foot}%
    \usebeamerfont{title in head/foot}\insertshorttitle
  \end{beamercolorbox}}%
  \vskip0pt%
}

%creating a rectangle box def
\newtcbox{\mybox}[1][red]{arc=0pt,outer arc=0pt,colback=#1!10!white,colframe=#1!50!black, boxsep=0pt,left=1pt,right=1pt,top=2pt,bottom=2pt,boxrule=0pt,bottomrule=1pt,toprule=1pt}

\newtcbox{\xmybox}[1][red]{arc=7pt,colback=#1!10!white,colframe=#1!50!black,before upper={\rule[-3pt]{0pt}{10pt}},boxrule=1pt,boxsep=0pt,left=6pt,right=6pt,top=2pt,bottom=2pt}
%the ``on line'' option doesn't work. so omitting it

%===== spacing =====

\def\extraspacing{\vspace{2mm} \noindent}
\def\vgap{\vspace{2mm}}
\def\hgap{\textrm{\hspace{1mm}}}

%===== tabbing =====

\def\tab{\hspace{2mm}}
\def\tabpos{\hspace{4mm} \= \hspace{4mm} \= \hspace{4mm} \= \hspace{4mm} \=
\hspace{4mm} \= \hspace{4mm} \= \hspace{4mm} \= \hspace{4mm} \= \hspace{4mm}
\kill}
\newcommand{\mytab}[1]{\begin{tabbing}\tabpos #1\end{tabbing}}

%===== blocks =====

% \newtheorem{theorem}{Theorem}
% \newtheorem{lemma}{Lemma}
% \newtheorem{corollary}{Corollary}
% \newtheorem{proposition}{Proposition}
% \newtheorem{definition}{Definition}
% \newtheorem{problem}{Problem}

\newcommand{\cbox}[2]{\begin{tcolorbox}[arc=0mm, colframe=#1!50!black, colback=#1!10!white]#2\end{tcolorbox}}
\newcommand{\minipg}[2]{\begin{center}\begin{minipage}{#1}#2\end{minipage}\end{center}}
\newcommand{\myfrm}[1]{\begin{frame}\begin{small}#1\end{small}\end{frame}} 
\newcommand{\myitems}[1]{\begin{itemize}#1\end{itemize}}
\newcommand{\myenums}[1]{\begin{enumerate}#1\end{enumerate}}
\newcommand{\myfig}[1]{\begin{figure}\centering #1\end{figure}}
    
%===== math macros =====
\newcommand{\bm}[1]{\textrm{\boldmath${#1}$}}
%\newcommand{\smat}[2]{\left[\begin{tabular}{#1}#2\end{tabular}\right]}
%\newcommand{\bmat}[2]{\left|\begin{tabular}{#1}#2\end{tabular}\right|}
\newcommand{\bmat}[1]{\begin{bmatrix}#1\end{bmatrix}}
\newcommand{\vmat}[1]{\begin{vmatrix}#1\end{vmatrix}}
\newcommand{\myeqn}[1]{\begin{eqnarray}#1\end{eqnarray}}
\newcommand{\set}[1]{\{#1\}}

\def\eps{\epsilon}
\def\fr{\frac}
\def\lc{\lceil}
\def\lf{\lfloor}
\def\rc{\rceil}
\def\rf{\rfloor}
\def\Pr{\textrm{\boldmath$Pr$}}
\def\expt{\textrm{\boldmath$E$}}
\def\real{\mathbb{R}}
\def\int{\mathbb{Z}}
\def\*{\star}
\def\tO{\tilde{O}}

\DeclareMathOperator*{\argmin}{arg\,min}
\DeclareMathOperator*{\polylg}{polylg}
\DeclareMathOperator*{\polylog}{polylog}
\DeclareMathOperator*{\intr}{\cap}

\def\nn{\nonumber}
\def\mit{\mathit}


%===== misc =====

\def\done{\hspace*{\fill} $\framebox[2mm]{}$}	% end of proof
\def\ttt{\texttt}

%===== coloring =====
\newcommand{\red}[1]{\textcolor{red}{#1}}
\newcommand{\bred}[1]{\textcolor{red}{\bf #1}}
\newcommand{\blue}[1]{\textcolor{blue}{\bf #1}}

\usepackage{color}
\usepackage{graphicx}
\usepackage{multirow}
\usepackage{wrapfig}
\usepackage[skins,breakable]{tcolorbox}

\def\done{\hfill$\square$}
\def\ttt{\texttt}
\def\vgap{\vspace{5mm}}

\def\abt{\ttt{ABORT}}
\def\best{\mit{best}}
\def\cmt{\ttt{COMMIT}}
\def\ins{\ttt{INSERT}}
\def\rd{\ttt{READ}}
\def\size{\mit{size}}
\def\sort{\mit{sort}}
\def\wt{\ttt{WRITE}}

\title[DATABASE SYSTEM PRINCIPLES]{Transactions 4:\\ Locking}

\author[Yufei Tao @ NTU]{Yufei Tao}
\institute[]{\url{https://www.cse.cuhk.edu.hk/~taoyf}}
\date{}

% \def\dtm{\mathit{d\mbox{-}tm}}
% \def\ftm{\mathit{f\mbox{-}tm}}
\def\bestext{\mathit{best\mbox{-}ext}}

\begin{document}
%-------------------------------------------------------------
\begin{frame}
    \titlepage
%     \begin{tcolorbox}[arc=0mm, colframe=green!50!black, colback=green!10!white] 
%     \end{tcolorbox}
\end{frame}
%-------------------------------------------------------------
\begin{frame}
\begin{small}
    Recall that the \bred{serializable} isolation level should be enforced for critical transactions (such as money transfers). This lecture will discuss how to guarantee this isolation level with \bred{locking}.
\end{small}    
\end{frame}
%-------------------------------------------------------------
\myfrm{
    \cbox{blue}{
        Transactions can only see themselves.
    }

    \vgap

    \cbox{green}{

    \blue{Example:} Suppose that we have two \blue{active} transactions (i.e., transactions in progress). \\[2mm]

    \begin{tabular}{cc}
            \begin{minipage}{0.47\linewidth}
                \begin{center}
                \begin{tabular}{c}
                    $\red{T_1}$ \\
                    \hline
                    \rd($A$) \\
                    $A = A + 1$ \\
                    \wt($A$) \\
                    \cmt
                \end{tabular}
                \end{center}
            \end{minipage}
            &
            \begin{minipage}{0.47\linewidth}
                \begin{center}
                \begin{tabular}{c}
                    $\red{T_2}$ \\
                    \hline
                    \rd($A$) \\
                    $A = A + 1$ \\
                    \wt($A$) \\
                    \cmt
                \end{tabular}
                \end{center}
            \end{minipage}
        \end{tabular}

    \vgap

    These transactions are not aware of each other.

    }


}
%-------------------------------------------------------------
\myfrm{
    \cbox{blue}{
        Database systems, which are \bred{multi-tasking} systems, execute the instructions of the active transactions in an interleaving manner. The mission of a \blue{concurrency control protocol} is to decide whether it is safe to execute an instruction without violating the target isolation level.
    }

    \vgap

    Today, we consider that \bred{serializable} is the target isolation level.

    \vgap

    \cbox{yellow}{
        We will discuss protocols based on a 3-way locking mechanism.
    }
}
%-------------------------------------------------------------
\myfrm{
    \xmybox{Rules of 3-Way Locking}

    \vgap

    \cbox{blue}{
        \blue{Rule 1:} Three locks: \bred{S} (\blue{shared}), \bred{U} (\blue{update}), and \bred{X} (\blue{exclusive}). Each lock applies to a database value (i.e., the value of a tuple under an attribute).
    }

    \cbox{blue}{
        \blue{Rule 2:} Before a transaction can \bred{read} a database value $\red{A}$, it must be holding an \bred{S- or a U-lock} on $A$. Before a transaction can \bred{write} a database value $\red{A}$, it must be holding an \bred{X-lock} on $A$.
    }


}
%-------------------------------------------------------------
\myfrm{
    \xmybox{Rules of 3-Way Locking}

    %\vgap

    \cbox{blue}{
        \blue{Rule 3:} (Compatibility Rules)
        \myitems{
            \item An S-lock on a value $\red{A}$ is \blue{compatible} with another S- or a U-lock on $A$.
            \item A U-lock on a value $\red{A}$ is \blue{compatible} with an S- or another U-lock on $A$.
        }

        An X-lock on a value $\red{A}$ is \bred{not compatible} with any other lock on $A$, regardless of its type.
    }

    \cbox{blue}{
        \blue{Rule 4:} To obtain a lock on a value $\red{A}$, a transaction needs to make a request. The database grants the request only if the request is \bred{compatible with the existing locks on $A$}.

        \vgap

        If granted, the transaction proceeds; otherwise (i.e., denied), the transaction is suspended until its request is granted.
    }
}
%-------------------------------------------------------------
\myfrm{
    \xmybox{Rules of 3-Way Locking}

    %\vgap

    \cbox{blue}{
        \blue{Rule 5:} A transaction releases a lock by making an \blue{unlock} request, which is always granted by the database.
    }

    \cbox{blue}{
        \blue{Rule 6:} If a transaction is holding a U-lock on a value $\red{A}$, it can issue an \blue{upgrade} request to acquire an X-clock on $A$. The database grants the request only if \bred{no other S- or U-locks exist on $A$}.

        \vgap

        If granted, the transaction releases the U-lock on $A$ and proceeds. Otherwise (i.e., denied), the transaction is suspended while holding the U-lock, until its request is granted.
    }
}
%-------------------------------------------------------------

\end{document} 



