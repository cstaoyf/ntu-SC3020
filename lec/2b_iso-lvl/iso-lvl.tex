\documentclass{beamer}

\usetheme{Warsaw}
%\usetheme{CambridgeUS}

% modification history
% created on 18 sep 2011
% modified on 25 mar

%\usepackage{amsfonts, amsmath, amssymb}

%\setbeamertemplate{theorems}[numbered]
%\setbeamertemplate{theorems}[ams style] 
\usepackage[skins,breakable]{tcolorbox}
%\usepackage[normalem]{ulem}

%\usefonttheme[onlymath]{serif}                     // change the style of math font 

%=============set slide number=================
\addtobeamertemplate{navigation symbols}{}{
    \usebeamerfont{footline}
    \usebeamercolor[fg]{footline}
    \hspace{1em}
    \insertframenumber/\inserttotalframenumber
}
\setbeamercolor{footline}{fg=black}
\setbeamerfont{footline}{series=\bfseries}


%=============set footline=====================
\setbeamertemplate{footline}
{
  \leavevmode%
  \hbox{%
  \begin{beamercolorbox}[wd=.55\paperwidth,ht=2.25ex,dp=1ex,center]{author in head/foot}%
    \usebeamerfont{author in head/foot}\insertshortauthor
  \end{beamercolorbox}%
  \begin{beamercolorbox}[wd=.45\paperwidth,ht=2.25ex,dp=1ex,center]{title in head/foot}%
    \usebeamerfont{title in head/foot}\insertshorttitle
  \end{beamercolorbox}}%
  \vskip0pt%
}

%creating a rectangle box def
\newtcbox{\mybox}[1][red]{arc=0pt,outer arc=0pt,colback=#1!10!white,colframe=#1!50!black, boxsep=0pt,left=1pt,right=1pt,top=2pt,bottom=2pt,boxrule=0pt,bottomrule=1pt,toprule=1pt}

\newtcbox{\xmybox}[1][red]{arc=7pt,colback=#1!10!white,colframe=#1!50!black,before upper={\rule[-3pt]{0pt}{10pt}},boxrule=1pt,boxsep=0pt,left=6pt,right=6pt,top=2pt,bottom=2pt}
%the ``on line'' option doesn't work. so omitting it

%===== spacing =====

\def\extraspacing{\vspace{2mm} \noindent}
\def\vgap{\vspace{2mm}}
\def\hgap{\textrm{\hspace{1mm}}}

%===== tabbing =====

\def\tab{\hspace{2mm}}
\def\tabpos{\hspace{4mm} \= \hspace{4mm} \= \hspace{4mm} \= \hspace{4mm} \=
\hspace{4mm} \= \hspace{4mm} \= \hspace{4mm} \= \hspace{4mm} \= \hspace{4mm}
\kill}
\newcommand{\mytab}[1]{\begin{tabbing}\tabpos #1\end{tabbing}}

%===== blocks =====

% \newtheorem{theorem}{Theorem}
% \newtheorem{lemma}{Lemma}
% \newtheorem{corollary}{Corollary}
% \newtheorem{proposition}{Proposition}
% \newtheorem{definition}{Definition}
% \newtheorem{problem}{Problem}

\newcommand{\cbox}[2]{\begin{tcolorbox}[arc=0mm, colframe=#1!50!black, colback=#1!10!white]#2\end{tcolorbox}}
\newcommand{\minipg}[2]{\begin{center}\begin{minipage}{#1}#2\end{minipage}\end{center}}
\newcommand{\myfrm}[1]{\begin{frame}\begin{small}#1\end{small}\end{frame}} 
\newcommand{\myitems}[1]{\begin{itemize}#1\end{itemize}}
\newcommand{\myenums}[1]{\begin{enumerate}#1\end{enumerate}}
\newcommand{\myfig}[1]{\begin{figure}\centering #1\end{figure}}
    
%===== math macros =====
\newcommand{\bm}[1]{\textrm{\boldmath${#1}$}}
%\newcommand{\smat}[2]{\left[\begin{tabular}{#1}#2\end{tabular}\right]}
%\newcommand{\bmat}[2]{\left|\begin{tabular}{#1}#2\end{tabular}\right|}
\newcommand{\bmat}[1]{\begin{bmatrix}#1\end{bmatrix}}
\newcommand{\vmat}[1]{\begin{vmatrix}#1\end{vmatrix}}
\newcommand{\myeqn}[1]{\begin{eqnarray}#1\end{eqnarray}}
\newcommand{\set}[1]{\{#1\}}

\def\eps{\epsilon}
\def\fr{\frac}
\def\lc{\lceil}
\def\lf{\lfloor}
\def\rc{\rceil}
\def\rf{\rfloor}
\def\Pr{\textrm{\boldmath$Pr$}}
\def\expt{\textrm{\boldmath$E$}}
\def\real{\mathbb{R}}
\def\int{\mathbb{Z}}
\def\*{\star}
\def\tO{\tilde{O}}

\DeclareMathOperator*{\argmin}{arg\,min}
\DeclareMathOperator*{\polylg}{polylg}
\DeclareMathOperator*{\polylog}{polylog}
\DeclareMathOperator*{\intr}{\cap}

\def\nn{\nonumber}
\def\mit{\mathit}


%===== misc =====

\def\done{\hspace*{\fill} $\framebox[2mm]{}$}	% end of proof
\def\ttt{\texttt}

%===== coloring =====
\newcommand{\red}[1]{\textcolor{red}{#1}}
\newcommand{\bred}[1]{\textcolor{red}{\bf #1}}
\newcommand{\blue}[1]{\textcolor{blue}{\bf #1}}

\usepackage{color}
\usepackage{graphicx}
\usepackage{multirow}
\usepackage{wrapfig}
\usepackage[skins,breakable]{tcolorbox}

\def\done{\hfill$\square$}
\def\ttt{\texttt}
\def\vgap{\vspace{5mm}}

\def\abt{\ttt{ABORT}}
\def\best{\mit{best}}
\def\cmt{\ttt{COMMIT}}
\def\ins{\ttt{INSERT}}
\def\rd{\ttt{READ}}
\def\size{\mit{size}}
\def\sort{\mit{sort}}
\def\wt{\ttt{WRITE}}

\title[DATABASE SYSTEM PRINCIPLES]{Transactions 2:\\ Isolation Levels}

\author[Yufei Tao @ NTU]{Yufei Tao}
\institute[]{\url{https://www.cse.cuhk.edu.hk/~taoyf}}
\date{}

% \def\dtm{\mathit{d\mbox{-}tm}}
% \def\ftm{\mathit{f\mbox{-}tm}}
\def\bestext{\mathit{best\mbox{-}ext}}

\begin{document}
%-------------------------------------------------------------
\begin{frame}
    \titlepage
%     \begin{tcolorbox}[arc=0mm, colframe=green!50!black, colback=green!10!white] 
%     \end{tcolorbox}
\end{frame}
%-------------------------------------------------------------
\begin{frame}
\begin{small}
    This lecture focuses on \bred{concurrent schedules} (where the instructions of different transactions are interleaved). We will see that they can cause different types of ``strange'' phenomena --- called \blue{anomalies} --- that can never arise in serial schedules. A database system may choose to allow \bred{none or some} of those anomalies. Such a policy is formalized by specifying an appropriate \blue{isolation level}.
    %\vgap
\end{small}    
\end{frame}
%-------------------------------------------------------------
\myfrm{
    \cbox{yellow}{
    \centering
        Anomaly I: Non-Serializability
    }
}
%-------------------------------------------------------------
\myfrm{
    \cbox{green}{
        \blue{Example:} We saw the following non-serializable schedule earlier.

        \begin{center}
        \begin{tabular}{c|c}
            $\red{T_1}$ & $\red{T_2}$\\
            \hline
            \rd($A$) & \\
            $A = A + 1$ &\\
            & \rd($A$) \\
            & $A = A + 1$ \\
            \wt($A$) & \\
            & \wt($A$) \\
            & \cmt \\
            \cmt &
        \end{tabular}
        \end{center}

        \vgap

        If $A$ was 0 before running the schedule, what is its value afterwards?
    }
}
%-------------------------------------------------------------
\myfrm{
    \cbox{yellow}{
    \centering
        Anomaly II: Phantoms
    }
}
%-------------------------------------------------------------
\myfrm{
    \cbox{blue}{
        A \blue{phantom} refers to the situation where the same query, executed twice in the same transaction, returns different results because one or more tuples are inserted by another transaction.
    }

    \cbox{green}{
        \blue{Example:} Relation $R(X, Y)$.

        \begin{center}
        \begin{tabular}{c|c}
            $\red{T_1}$ & $\red{T_2}$\\
            \hline
            find all $t \in R$ with $\red{t.X = 1}$ & \\
            & $\ins$ tuple $(1, y)$ in $R$ \\
            find all $t \in R$ with $\red{t.X = 1}$ & \\
            & \cmt \\
            \cmt &
        \end{tabular}
        \end{center}
    }

%     \cbox{blue}{
%         Phantoms can be regarded as a form of non-serializability.
%     }
}
%-------------------------------------------------------------
\myfrm{
    \cbox{yellow}{
    \centering
        Anomaly III: Non-Repeatable Reads
    }
}
%-------------------------------------------------------------
\myfrm{
    \cbox{blue}{
        A \blue{non-repeatable read} situation occurs when two identical reads retrieve different information even though the transaction performs no modifications on the relevant data.
    }

    \cbox{green}{
        \blue{Example:}
        \vspace{-5mm}

        \begin{center}
        \begin{tabular}{c|c}
            $\red{T_1}$ & $\red{T_2}$\\
            \hline
            & \rd($A$) \\
            \rd($A$) & \\
            $A = A + 1$ &\\
            \wt($A$) & \\
            \cmt & \\
            & \rd($A$) \\
            & \cmt
        \end{tabular}
        \end{center}

        The second $\rd(A)$ of $T_1$ fetches a value of $A$ different from the one fetched by the first $\rd(A)$.
    }
%     \cbox{blue}{
%         Non-repeatable reads can be regarded as a form of non-serializability.
%     }
}
%-------------------------------------------------------------
\myfrm{
    \cbox{yellow}{
    \centering
        Anomaly IV: Dirty Reads
    }
}
%-------------------------------------------------------------
\myfrm{
    \cbox{blue}{
        A read operation is \blue{dirty} if it reads a value that was written by another transaction that has not committed.
    }

    \cbox{green}{
        \blue{Example:}
        \begin{center}
        \begin{tabular}{c|c}
            $\red{T_1}$ & $\red{T_2}$\\
            \hline
            \rd($A$) & \\
            $A = A + 1$ &\\
            \wt($A$) & \\
            & \rd($A$) \\
%             & $A = A + 1$ \\
%             & \wt($A$) \\
            \cmt & \\
            & \cmt
        \end{tabular}
        \end{center}

        The read performed by $T_2$ is dirty.
    }
}
%-------------------------------------------------------------

\myfrm{
    \cbox{red}{
        All the above anomalies are bad.
    }
    \cbox{blue}{
        However, preventing them all may severely reduce the concurrency level (forcing all schedules to be serial in the worst case). A database system often allows each transaction to specify a so-called \bred{isolation level}, which indicates what anomalies can be tolerated.
    }
}
%-------------------------------------------------------------
\myfrm{
    \cbox{yellow}{
        \centering
        ANSI Isolation Levels
    }
    Note: ANSI = American National Standards Institute
}
%-------------------------------------------------------------
\myfrm{
    \xmybox{Isolation Levels}

    \vgap

    \begin{tabular}{c||c|c|c}
        \hspace{-5mm} \bred{isolation level} & dirty reads & unrepeatable reads & non-serializability
        \\
        \hline\hline
        \hspace{-5mm} \blue{read uncommited} & yes & yes & yes \\
        \hline
        \hspace{-5mm} \blue{read commited} & no & yes & yes \\
        \hline
        \hspace{-5mm} \blue{repeatable reads} & no & no & yes \\
        \hline
        \hspace{-5mm} \blue{serializable} & no & no & no
    \end{tabular}


    %\vgap

    \cbox{red}{
        Phantoms may occur at the read-uncommited, read-commited, and repeatable-reads levels, but \bred{not} at the serializable level.
    }
    \cbox{blue}{
        \blue{Rule of Thumb:}
        \myitems{
            \item The \bred{serializable level} should be enforced for critical transactions (e.g., money transfers).
            \item The \bred{read-committed} level (which is the default level at ORACLE, a popular database system) may suffice for non-critical transactions.
        }

    }

}
%-------------------------------------------------------------
\end{document} 



