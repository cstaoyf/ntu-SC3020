\documentclass{beamer}

\usetheme{Warsaw}
%\usetheme{CambridgeUS}

% modification history
% created on 18 sep 2011
% modified on 25 mar

%\usepackage{amsfonts, amsmath, amssymb}

%\setbeamertemplate{theorems}[numbered]
%\setbeamertemplate{theorems}[ams style] 
\usepackage[skins,breakable]{tcolorbox}
%\usepackage[normalem]{ulem}

%\usefonttheme[onlymath]{serif}                     // change the style of math font 

%=============set slide number=================
\addtobeamertemplate{navigation symbols}{}{
    \usebeamerfont{footline}
    \usebeamercolor[fg]{footline}
    \hspace{1em}
    \insertframenumber/\inserttotalframenumber
}
\setbeamercolor{footline}{fg=black}
\setbeamerfont{footline}{series=\bfseries}


%=============set footline=====================
\setbeamertemplate{footline}
{
  \leavevmode%
  \hbox{%
  \begin{beamercolorbox}[wd=.55\paperwidth,ht=2.25ex,dp=1ex,center]{author in head/foot}%
    \usebeamerfont{author in head/foot}\insertshortauthor
  \end{beamercolorbox}%
  \begin{beamercolorbox}[wd=.45\paperwidth,ht=2.25ex,dp=1ex,center]{title in head/foot}%
    \usebeamerfont{title in head/foot}\insertshorttitle
  \end{beamercolorbox}}%
  \vskip0pt%
}

%creating a rectangle box def
\newtcbox{\mybox}[1][red]{arc=0pt,outer arc=0pt,colback=#1!10!white,colframe=#1!50!black, boxsep=0pt,left=1pt,right=1pt,top=2pt,bottom=2pt,boxrule=0pt,bottomrule=1pt,toprule=1pt}

\newtcbox{\xmybox}[1][red]{arc=7pt,colback=#1!10!white,colframe=#1!50!black,before upper={\rule[-3pt]{0pt}{10pt}},boxrule=1pt,boxsep=0pt,left=6pt,right=6pt,top=2pt,bottom=2pt}
%the ``on line'' option doesn't work. so omitting it

%===== spacing =====

\def\extraspacing{\vspace{2mm} \noindent}
\def\vgap{\vspace{2mm}}
\def\hgap{\textrm{\hspace{1mm}}}

%===== tabbing =====

\def\tab{\hspace{2mm}}
\def\tabpos{\hspace{4mm} \= \hspace{4mm} \= \hspace{4mm} \= \hspace{4mm} \=
\hspace{4mm} \= \hspace{4mm} \= \hspace{4mm} \= \hspace{4mm} \= \hspace{4mm}
\kill}
\newcommand{\mytab}[1]{\begin{tabbing}\tabpos #1\end{tabbing}}

%===== blocks =====

% \newtheorem{theorem}{Theorem}
% \newtheorem{lemma}{Lemma}
% \newtheorem{corollary}{Corollary}
% \newtheorem{proposition}{Proposition}
% \newtheorem{definition}{Definition}
% \newtheorem{problem}{Problem}

\newcommand{\cbox}[2]{\begin{tcolorbox}[arc=0mm, colframe=#1!50!black, colback=#1!10!white]#2\end{tcolorbox}}
\newcommand{\minipg}[2]{\begin{center}\begin{minipage}{#1}#2\end{minipage}\end{center}}
\newcommand{\myfrm}[1]{\begin{frame}\begin{small}#1\end{small}\end{frame}} 
\newcommand{\myitems}[1]{\begin{itemize}#1\end{itemize}}
\newcommand{\myenums}[1]{\begin{enumerate}#1\end{enumerate}}
\newcommand{\myfig}[1]{\begin{figure}\centering #1\end{figure}}
    
%===== math macros =====
\newcommand{\bm}[1]{\textrm{\boldmath${#1}$}}
%\newcommand{\smat}[2]{\left[\begin{tabular}{#1}#2\end{tabular}\right]}
%\newcommand{\bmat}[2]{\left|\begin{tabular}{#1}#2\end{tabular}\right|}
\newcommand{\bmat}[1]{\begin{bmatrix}#1\end{bmatrix}}
\newcommand{\vmat}[1]{\begin{vmatrix}#1\end{vmatrix}}
\newcommand{\myeqn}[1]{\begin{eqnarray}#1\end{eqnarray}}
\newcommand{\set}[1]{\{#1\}}

\def\eps{\epsilon}
\def\fr{\frac}
\def\lc{\lceil}
\def\lf{\lfloor}
\def\rc{\rceil}
\def\rf{\rfloor}
\def\Pr{\textrm{\boldmath$Pr$}}
\def\expt{\textrm{\boldmath$E$}}
\def\real{\mathbb{R}}
\def\int{\mathbb{Z}}
\def\*{\star}
\def\tO{\tilde{O}}

\DeclareMathOperator*{\argmin}{arg\,min}
\DeclareMathOperator*{\polylg}{polylg}
\DeclareMathOperator*{\polylog}{polylog}
\DeclareMathOperator*{\intr}{\cap}

\def\nn{\nonumber}
\def\mit{\mathit}


%===== misc =====

\def\done{\hspace*{\fill} $\framebox[2mm]{}$}	% end of proof
\def\ttt{\texttt}

%===== coloring =====
\newcommand{\red}[1]{\textcolor{red}{#1}}
\newcommand{\bred}[1]{\textcolor{red}{\bf #1}}
\newcommand{\blue}[1]{\textcolor{blue}{\bf #1}}

\usepackage{color}
\usepackage{graphicx}
\usepackage{multirow}
\usepackage{wrapfig}
\usepackage[skins,breakable]{tcolorbox}

\def\done{\hfill$\square$}
\def\ttt{\texttt}
\def\vgap{\vspace{5mm}}

\def\best{\mit{best}}
\def\size{\mit{size}}
\def\sort{\mit{sort}}

\title[DATABASE SYSTEM PRINCIPLES]{Query Processing 6:\\ Join Order Optimization}

\author[Yufei Tao @ NTU]{Yufei Tao}
\institute[]{\url{https://www.cse.cuhk.edu.hk/~taoyf}}
\date{}

% \def\dtm{\mathit{d\mbox{-}tm}}
% \def\ftm{\mathit{f\mbox{-}tm}}
\def\bestext{\mathit{best\mbox{-}ext}}

\begin{document}
%-------------------------------------------------------------
\begin{frame}
    \titlepage
%     \begin{tcolorbox}[arc=0mm, colframe=green!50!black, colback=green!10!white] 
%     \end{tcolorbox}
\end{frame}
%-------------------------------------------------------------
\begin{frame}
\begin{small}
    In this lecture, we will discuss the \blue{join order} problem, which is the most important query rewriting optimization in database systems.

    \vgap

    Recall:
    \cbox{blue}{
        \blue{Query rewriting} converts the original query to an equivalent query using laws of relational algebra.
    }
    %\vgap
\end{small}    
\end{frame}
%-------------------------------------------------------------
\myfrm{
    \xmybox{Laws of natural joins}

    \vgap

    Let $\red{R_1}, \red{R_2}$, and $\red{R_3}$ be relations.

    \vgap

    \blue{Commutativity:} $R_1 \bowtie R_2 = R_2 \bowtie R_1$ \\
    \blue{Associativity:} $R_1 \bowtie (R_2 \bowtie R_3) = (R_1 \bowtie R_2) \bowtie R_3$ \\

    \vgap

    Therefore: $(R_1 \bowtie R_2) \bowtie R_3 = (R_1 \bowtie R_3) \bowtie R_2 = (R_2 \bowtie R_3) \bowtie R_1$.

    \vgap

    \cbox{blue}{
        In general, we can compute the natural join of several relations by ordering the relations arbitrarily.
    }
}
%-------------------------------------------------------------
\myfrm{
    \cbox{blue}{
        Various join orderings can have extremely different I/O costs.
    }

    \cbox{green}{
        \blue{Example:}\\
        Relations $\red{R_1(A,B)}$, $\red{R_2(C,D)}$, $\red{R_3(A,D)}$, each having 1000 tuples, where $A$ is the primary key of $R_1$.

        \vgap

        \blue{Join order 1:} $R_1 \bowtie R_2 \bowtie R_3$ \\
        The intermediate relation $\red{R_4} = R_1 \bowtie R_2$ has $1000^2 = 10^6$ tuples!

        \vgap

        \blue{Join order 2:} $R_1 \bowtie R_3 \bowtie R_2$ \\
        The intermediate relation $\red{R_5} = R_1 \bowtie R_3$ has at most $1000$ tuples.

        %\vgap

        \cbox{blue}{
        Join order 1 is expected to be significantly more expensive.}
    }
}
%-------------------------------------------------------------
\myfrm{
    %In general, let $\red{R_1}, \red{R_2}, ..., \red{R_k}$ be $\red{k}$ relations.

    \cbox{blue}{
        Let $\red{S}$ be a set of relations. An \blue{ordering} of $S$ is a permutation of the relations in $S$.
    }
    %Number of orderings $= k!$.

%    \vgap

    \cbox{blue}{
        Fix an arbitrary ordering $\red{\pi} = (R_1, R_2, ..., R_t)$ where $t \ge 2$.
        \myitems{
            \item For each $\red{i} \in [2, t]$, $\red{R_1 \bowtie R_2 \bowtie ... \bowtie R_i}$ is a \blue{prefix} of $\pi$. The \blue{join size} \bred{of the prefix} is $|R_1 \bowtie R_2 \bowtie ... \bowtie R_i|$, i.e., the number of tuples output by the join defined by the prefix.

            \item The \blue{penalty} \bred{of $\pi$} is the sum of the join sizes of its $t-1$ prefixes.
        }
    }

    \cbox{blue}{
        \blue{The Join Order Problem:} Given $\red{k}$ relations $\red{R_1}, \red{R_2}, ..., \red{R_k}$, find an ordering of $\set{R_1, .., R_k}$ with the smallest penalty.
    }
}
%-------------------------------------------------------------
% \myfrm{
%     \cbox{green}{
%         \blue{Example:}
%         Relations $\red{R_1(A,B)}$, $\red{R_2(C,D)}$, $\red{R_3(A,D)}$. \\
%
%         %\vgap
%
%         \begin{tabular}{cc}
%             \begin{minipage}{0.45\linewidth}
%                 There are 6 join orders:
%                 \myenums{
%                     \item $R_1 \bowtie R_2 \bowtie R_3$ \\[-1mm]
%                     \item $R_1 \bowtie R_3 \bowtie R_2$ \\[-1mm]
%                     \item $R_2 \bowtie R_1 \bowtie R_3$ \\[-1mm]
%                     \item $R_2 \bowtie R_3 \bowtie R_1$ \\[-1mm]
%                     \item $R_3 \bowtie R_1 \bowtie R_2$ \\[-1mm]
%                     \item $R_3 \bowtie R_2 \bowtie R_1$
%                 }
%                 Each join order corresponding to an ordering of $\set{R_1, R_2, R_3}$.
%             \end{minipage}
%             &
%             \begin{minipage}{0.45\linewidth}
%                 The penalty of the ordering $(R_2, R_1, R_3)$ equals the sum of $|R_2 \bowtie R_1|$ and $|R_2 \bowtie R_1 \bowtie R_3|$.
%             \end{minipage}
%         \end{tabular}
%     }
% }
%-------------------------------------------------------------
\myfrm{
    \cbox{blue}{
        Each ordering corresponds to a ``left-deep'' query plan. The \bred{join size of a prefix} is the output size of a join operation in the plan. The  \bred{penalty of the ordering} is the total output size of all the joins in the query plan. The join order problem essentially aims to minimize the total output size of the \bred{intermediate} joins (i.e., ``non-root'' joins in the plan). (\blue{Think}: why?).
    }
    \cbox{green}{
        \blue{Example}

        \vspace{-5mm}
        \begin{center}
            \includegraphics[height=30mm]{./artwork/ld1}
        \end{center}
        \vspace{-2mm}

        Consider the ordering $(R_1, R_2, R_3, R_4)$. The join size of the prefix $R_1 \bowtie R_2$ is $|R_5|$ (see $R_5$ in the figure), and that of the prefix $R_1 \bowtie R_2 \bowtie R_3$ is $|R_6|$. Note that $R_5$ and $R_6$ are the outputs of the intermediate joins in the query plan on the left.
    }
}
%-------------------------------------------------------------
\myfrm{
    Next, we will discuss how to solve the join order problem by assuming a \blue{size oracle}, which returns the output size of any join.

    \vgap

    \cbox{red}{
        Size oracles do not exist in reality. In reality, given any join, a database system can compute an \bred{estimate} of its output size. These estimates serve the purpose of size oracle. We will discuss such estimates in a tutorial.
    }

    \vgap

    Recall that the input to the problem has $\red{k}$ relations $\red{R_1}, \red{R_2}, ..., \red{R_k}$. The problem can be solved naively with $\red{O(k \cdot k!)}$ calls to the size oracle (\blue{think}: how?).

    \vgap

    \cbox{blue}{
        We will show that the problem can be solved with $\red{O(k \cdot 2^k)}$ oracle calls by dynamic programming.
    }
}
%-------------------------------------------------------------
\myfrm{
    Recall that $R_1, R_2, ..., R_k$ are the input to the join order problem.

    \vgap

    Let $\red{S}$ be a subset of $\set{R_1, R_2, ..., R_k}$ with size $|S| \ge 2$.

    \cbox{blue}{
        Define
        \myitems{
            \item $\red{\size(S)}$ as the size of the join involving all the relations in $S$;
            \item $\red{\best(S)}$ as the smallest penalty of all orderings of $S$.
        }
    }


}
%-------------------------------------------------------------
\myfrm{
    \cbox{green}{
        \blue{Example:} Consider $k = 4$, i.e., the input is $R_1, R_2, ..., R_4$. Assume that the size oracle returns the values in the left table.

        \vgap

        \begin{tabular}{cc}
            \begin{minipage}{0.47\linewidth}
                \begin{tabular}{c|c}
                    $S$ & join size \\
                    \hline
                    $R_1,R_2$ & 1000 \\
                    $R_1,R_3$ & 100 \\
                    $R_1,R_4$ & 50 \\
                    $R_2,R_3$ & 20 \\
                    $R_2,R_4$ & 100 \\
                    $R_3,R_4$ & 30 \\
                    $R_1, R_2, R_3$ & 1000 \\
                    $R_1, R_2, R_4$ & 1500 \\
                    $R_1, R_3, R_4$ & 800 \\
                    $R_2, R_3, R_4$ & 70 \\
                    $R_1, R_2, R_3, R_4$ & 20
                \end{tabular}
            \end{minipage}
            &
            \begin{minipage}{0.5\linewidth}
                For $S = \set{R_1, R_3, R_4}$, we have
                \myitems{
                    \item $\size(S) = 800$
                    \item $\best(S) = 830$, which is the penalty of the ordering $(R_3, R_4, R_1)$.
                }

                \vgap

                For $S = \set{R_1, R_2}$, we have $\size(S)$ $= \best(S) = 1000$.
            \end{minipage}
        \end{tabular}

    }
}
%-------------------------------------------------------------
\myfrm{ \label{slide:thm}
    \cbox{blue}{
        \blue{Theorem 1:} Let $\red{S}$ be a subset of $\set{R_1, ..., R_k}$ with $|S| \ge 2$. \\ If $|S| = 2$, then
        \myeqn{
            \best(S) = \size(S). \label{thm:1}
        }
        If $|S| > 2$, then
        \myeqn{
            \best(S) = \size(S) + \min_{\substack{\text{subset $\red{S'} \in S$}\\ \text{with $\red{|S'| = |S| - 1}$}}} \best(S'). \label{thm:2}
        }
    }

    \cbox{green}{
        \blue{Example:} Continuing on the example of the previous slide, we know that $\best(\set{R_1, R_3, R_4})$ equals $\size(\set{R_1, R_3, R_4}) = 800$ plus the minimum of
        \myitems{
            \item $\best(\set{R_1, R_3}) = \size(\set{R_1, R_3}) = 100$
            \item $\best(\set{R_1, R_4}) = \size(\set{R_1, R_4}) = 50$
            \item $\best(\set{R_3, R_4}) = \size(\set{R_1, R_4}) = 30$.
        }
    }
}
%-------------------------------------------------------------
\myfrm{
    \cbox{blue}{
        \blue{Corollary 2:} Let $\red{S}$ be a subset of $\set{R_1, ..., R_k}$ with $|S| \ge 3$. If $\best(S')$ is available for every subset $S' \subseteq S$ with $|S'| = |S| - 1$, we can compute $\best(S)$ with $|S| \le k$ oracle calls.
    }

    The corollary naturally implies the following algorithm for solving the join order problem.

    \vgap

    \blue{Step 1:} Find $\best(S)$ for every $S \subseteq \set{R_1, ..., R_k}$ with $|S| = 2$ \\ (one oracle call for each $S$).

    \vgap

    \blue{Step $\red{t \ge 2}$:} Find $\best(S)$ for every $S \subseteq \set{R_1, ..., R_k}$ with $|S| = t+1$ \\
    (at most $k$ oracle calls for each $S$).

    \vgap

    We obtain $\best(\set{R_1, ..., R_k})$ after Step $k - 1$.
}
%-------------------------------------------------------------
\myfrm{
    \cbox{green}{
        \blue{Example:} Consider $k = 4$, i.e., the input is $R_1, R_2, ..., R_4$. Assume that the size oracle returns the values in the left table. The right table illustrates the computation of the $\best$ values.

        \vgap

        \begin{tabular}{cc}
            \hspace{-5mm}
            \begin{minipage}{0.47\linewidth}
                \begin{tabular}{c|c}
                    $S$ & join size \\
                    \hline
                    $R_1,R_2$ & 1000 \\
                    $R_1,R_3$ & 100 \\
                    $R_1,R_4$ & 50 \\
                    $R_2,R_3$ & 20 \\
                    $R_2,R_4$ & 100 \\
                    $R_3,R_4$ & 30 \\
                    $R_1, R_2, R_3$ & 1000 \\
                    $R_1, R_2, R_4$ & 1500 \\
                    $R_1, R_3, R_4$ & 800 \\
                    $R_2, R_3, R_4$ & 70 \\
                    $R_1, R_2, R_3, R_4$ & 20
                \end{tabular}
            \end{minipage}
            &
            \hspace{-8mm}
            \begin{minipage}{0.47\linewidth}
                \begin{tabular}{c|c|c}
                    $S$ & $\best$ & \# oracle calls\\
                    \hline
                    $R_1,R_2$ & 1000 & 1\\
                    $R_1,R_3$ & 100 & 1 \\
                    $R_1,R_4$ & 50 & 1 \\
                    $R_2,R_3$ & 20 & 1\\
                    $R_2,R_4$ & 100 & 1\\
                    $R_3,R_4$ & 30 & 1\\
                    $R_1, R_2, R_3$ & 1020 & 2 \\
                    $R_1, R_2, R_4$ & 1550 & 2\\
                    $R_1, R_3, R_4$ & 830 & 2 \\
                    $R_2, R_3, R_4$ & 90 & 2\\
                    $R_1, R_2, R_3, R_4$ & 110 & 3
                \end{tabular}
            \end{minipage}
        \end{tabular}

        \vspace{2mm}
        \blue{Optimal join order:} $R_2 \bowtie R_3 \bowtie R_4 \bowtie R_1$.
    }
}
%-------------------------------------------------------------
\myfrm{
    \xmybox{Analysis}

    \vgap

    $\set{R_1, R_2, ..., R_k}$ has less than $2^k$ subsets $S$ of size $|S| \ge 2$. The value $\best(S)$ is computed in one of the $k-1$ steps, and the computation requires at most $k$ oracle calls. The total number of oracle calls is therefore $O(k \cdot 2^k)$.
}
%-------------------------------------------------------------
\end{document} 



