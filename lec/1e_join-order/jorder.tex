\documentclass{beamer}

\usetheme{Warsaw}
%\usetheme{CambridgeUS}

% modification history
% created on 18 sep 2011
% modified on 25 mar

%\usepackage{amsfonts, amsmath, amssymb}

%\setbeamertemplate{theorems}[numbered]
%\setbeamertemplate{theorems}[ams style] 
\usepackage[skins,breakable]{tcolorbox}
%\usepackage[normalem]{ulem}

%\usefonttheme[onlymath]{serif}                     // change the style of math font 

%=============set slide number=================
\addtobeamertemplate{navigation symbols}{}{
    \usebeamerfont{footline}
    \usebeamercolor[fg]{footline}
    \hspace{1em}
    \insertframenumber/\inserttotalframenumber
}
\setbeamercolor{footline}{fg=black}
\setbeamerfont{footline}{series=\bfseries}


%=============set footline=====================
\setbeamertemplate{footline}
{
  \leavevmode%
  \hbox{%
  \begin{beamercolorbox}[wd=.55\paperwidth,ht=2.25ex,dp=1ex,center]{author in head/foot}%
    \usebeamerfont{author in head/foot}\insertshortauthor
  \end{beamercolorbox}%
  \begin{beamercolorbox}[wd=.45\paperwidth,ht=2.25ex,dp=1ex,center]{title in head/foot}%
    \usebeamerfont{title in head/foot}\insertshorttitle
  \end{beamercolorbox}}%
  \vskip0pt%
}

%creating a rectangle box def
\newtcbox{\mybox}[1][red]{arc=0pt,outer arc=0pt,colback=#1!10!white,colframe=#1!50!black, boxsep=0pt,left=1pt,right=1pt,top=2pt,bottom=2pt,boxrule=0pt,bottomrule=1pt,toprule=1pt}

\newtcbox{\xmybox}[1][red]{arc=7pt,colback=#1!10!white,colframe=#1!50!black,before upper={\rule[-3pt]{0pt}{10pt}},boxrule=1pt,boxsep=0pt,left=6pt,right=6pt,top=2pt,bottom=2pt}
%the ``on line'' option doesn't work. so omitting it

%===== spacing =====

\def\extraspacing{\vspace{2mm} \noindent}
\def\vgap{\vspace{2mm}}
\def\hgap{\textrm{\hspace{1mm}}}

%===== tabbing =====

\def\tab{\hspace{2mm}}
\def\tabpos{\hspace{4mm} \= \hspace{4mm} \= \hspace{4mm} \= \hspace{4mm} \=
\hspace{4mm} \= \hspace{4mm} \= \hspace{4mm} \= \hspace{4mm} \= \hspace{4mm}
\kill}
\newcommand{\mytab}[1]{\begin{tabbing}\tabpos #1\end{tabbing}}

%===== blocks =====

% \newtheorem{theorem}{Theorem}
% \newtheorem{lemma}{Lemma}
% \newtheorem{corollary}{Corollary}
% \newtheorem{proposition}{Proposition}
% \newtheorem{definition}{Definition}
% \newtheorem{problem}{Problem}

\newcommand{\cbox}[2]{\begin{tcolorbox}[arc=0mm, colframe=#1!50!black, colback=#1!10!white]#2\end{tcolorbox}}
\newcommand{\minipg}[2]{\begin{center}\begin{minipage}{#1}#2\end{minipage}\end{center}}
\newcommand{\myfrm}[1]{\begin{frame}\begin{small}#1\end{small}\end{frame}} 
\newcommand{\myitems}[1]{\begin{itemize}#1\end{itemize}}
\newcommand{\myenums}[1]{\begin{enumerate}#1\end{enumerate}}
\newcommand{\myfig}[1]{\begin{figure}\centering #1\end{figure}}
    
%===== math macros =====
\newcommand{\bm}[1]{\textrm{\boldmath${#1}$}}
%\newcommand{\smat}[2]{\left[\begin{tabular}{#1}#2\end{tabular}\right]}
%\newcommand{\bmat}[2]{\left|\begin{tabular}{#1}#2\end{tabular}\right|}
\newcommand{\bmat}[1]{\begin{bmatrix}#1\end{bmatrix}}
\newcommand{\vmat}[1]{\begin{vmatrix}#1\end{vmatrix}}
\newcommand{\myeqn}[1]{\begin{eqnarray}#1\end{eqnarray}}
\newcommand{\set}[1]{\{#1\}}

\def\eps{\epsilon}
\def\fr{\frac}
\def\lc{\lceil}
\def\lf{\lfloor}
\def\rc{\rceil}
\def\rf{\rfloor}
\def\Pr{\textrm{\boldmath$Pr$}}
\def\expt{\textrm{\boldmath$E$}}
\def\real{\mathbb{R}}
\def\int{\mathbb{Z}}
\def\*{\star}
\def\tO{\tilde{O}}

\DeclareMathOperator*{\argmin}{arg\,min}
\DeclareMathOperator*{\polylg}{polylg}
\DeclareMathOperator*{\polylog}{polylog}
\DeclareMathOperator*{\intr}{\cap}

\def\nn{\nonumber}
\def\mit{\mathit}


%===== misc =====

\def\done{\hspace*{\fill} $\framebox[2mm]{}$}	% end of proof
\def\ttt{\texttt}

%===== coloring =====
\newcommand{\red}[1]{\textcolor{red}{#1}}
\newcommand{\bred}[1]{\textcolor{red}{\bf #1}}
\newcommand{\blue}[1]{\textcolor{blue}{\bf #1}}

\usepackage{color}
\usepackage{graphicx}
\usepackage{multirow}
\usepackage{wrapfig}
\usepackage[skins,breakable]{tcolorbox}

\def\done{\hfill$\square$}
\def\ttt{\texttt}
\def\vgap{\vspace{5mm}}

\def\sort{\mit{sort}}

\title[DATABASE SYSTEM PRINCIPLES]{Query Processing 6:\\ Join Order Optimization}

\author[Yufei Tao @ NTU]{Yufei Tao}
\institute[]{\url{https://www.cse.cuhk.edu.hk/~taoyf}}
\date{}

% \def\dtm{\mathit{d\mbox{-}tm}}
% \def\ftm{\mathit{f\mbox{-}tm}}
\def\bestext{\mathit{best\mbox{-}ext}}

\begin{document}
%-------------------------------------------------------------
\begin{frame}
    \titlepage
%     \begin{tcolorbox}[arc=0mm, colframe=green!50!black, colback=green!10!white] 
%     \end{tcolorbox}
\end{frame}
%-------------------------------------------------------------
\begin{frame}
\begin{small}
    In this lecture, we will discuss the \blue{join order} problem, which is the most important query rewriting optimization in database systems.

    \vgap

    Recall:
    \cbox{blue}{
        \blue{Query rewriting} converts the original query to an equivalent query using laws of relational algebra.
    }
    %\vgap
\end{small}    
\end{frame}
%-------------------------------------------------------------
\myfrm{
    \xmybox{Laws of natural joins}

    \vgap

    Let $\red{R_1}, \red{R_2}$, and $\red{R_3}$ be relations.

    \vgap

    \blue{Commutativity:} $R_1 \bowtie R_2 = R_2 \bowtie R_1$ \\
    \blue{Associativity:} $R_1 \bowtie (R_2 \bowtie R_3) = (R_1 \bowtie R_2) \bowtie R_3$ \\
}
%-------------------------------------------------------------
\myfrm{
    \cbox{blue}{
        Various join orderings can have extremely different I/O costs.
    }

    \cbox{green}{
        \blue{Example:}\\
        Relations $\red{R_1(A,B)}$, $\red{R_2(C,D)}$, $\red{R_3(A,D)}$, each having 1000 tuples, where $A$ is the primary key of $R_1$.

        \vgap

        \blue{Ordering 1:} $R_1 \bowtie R_2 \bowtie R_3$ \\
        The intermediate relation $\red{R_4} = R_1 \bowtie R_2$ has $1000^2 = 10^6$ tuples!

        \vgap

        \blue{Ordering 2:} $R_1 \bowtie R_3 \bowtie R_2$ \\
        The intermediate relation $\red{R_5} = R_1 \bowtie R_3$ has at most $1000$ tuples.

        %\vgap

        \cbox{blue}{
        Ordering 2 is significantly more efficient than ordering 1 --- you can estimate the I/O cost of using, e.g., the sort join algorithm to compute the join under each ordering.}
    }
}
%-------------------------------------------------------------
\myfrm{
    In general, let $\red{R_1}, \red{R_2}, ..., \red{R_k}$ be $\red{k}$ relations.

    \cbox{blue}{
        An \blue{ordering} is a permutation of the $k$ relations.
    }
    Number of orderings $= k!$.

    \vgap

    \cbox{blue}{
        Fix an arbitrary ordering $\red{\pi} = R_{\red{a_1}}$, $R_{\red{a_2}}$, ..., $R_{\red{a_k}}$.
        \myitems{
            \item For each $\red{i} \in [2, k-1]$, the join $\red{R_{a_1} \bowtie R_{a_2} \bowtie ... \bowtie R_{a_i}}$ is a \blue{prefix} of $\pi$. The \blue{capacity} of the prefix is $|R_{a_1} \bowtie R_{a_2} \bowtie ... \bowtie R_{a_i}|$, i.e., the number of tuples in the result of the prefix.

            \item The \blue{capacity} of $\pi$ is the sum of the capacities of all its $k-2$ prefixes.
        }
    }
}
%-------------------------------------------------------------
\end{document} 


