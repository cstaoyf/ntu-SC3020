%\documentclass{...}

%modification history
% 9 jul 2011
% 29 nov 2011

\documentclass[11pt, letterpaper]{article}
\usepackage[margin=1in]{geometry}

% \documentclass[11pt, letterpaper]{article}
% 
% % ----- margins -----
% 
% \topmargin -1.5cm         % read Lamport p.163
% \oddsidemargin -0.04cm    % read Lamport p.163
% \evensidemargin -0.04cm   % same as oddsidemargin but for left-hand pages
% 
% % ----- texts -----
% 
% \textwidth 16.59cm
% \textheight 21.94cm
% 
% % ----- indendts and spacing -----
% 
% \parskip 0pt            	% spacing between paragraphs
% %\renewcommand{\baselinestretch}{1.5}	% uncomment for 1.5 spacing
% 
% \parindent 7mm		      % leading space for paragraphs between lines
% 
% % ----- page # -----
% 
% %\pagestyle{empty}         % uncomment if don't want page numbers





\usepackage{amsfonts, amsmath, amssymb, amsthm}
\usepackage{comment}
\usepackage{graphicx}
\usepackage{ifthen}
\usepackage{latexsym}
%\usepackage{times}
\usepackage[normalem]{ulem}

%modification history
% 2012
% may 3
% aug 11

%=== use the these packages if they are not already in use ===
%\usepackage{amsfonts, amsmath, amssymb, amsthm}
%=============================================================

%===== fonts =====
\def\ttt{\texttt}
%===== spacing =====

\def\extraspacing{\vspace{5mm} \noindent}
\def\figcapup{\vspace{-1mm}}
\def\figcapdown{\vspace{-0mm}}
\def\hgap{\textrm{\hspace{1mm}}}
\def\thmvgap{\vspace{0mm}}
\def\vgap{\vspace{2mm}}


%===== tabbing =====

\def\tab{\hspace{3mm}}
\def\tabpos{\hspace{4mm} \= \hspace{4mm} \= \hspace{4mm} \= \hspace{4mm} \=
\hspace{4mm} \= \hspace{4mm} \= \hspace{4mm} \= \hspace{4mm} \= \hspace{4mm}
\kill}

%===== blocks =====

\newtheorem{theorem}{Theorem}
\newtheorem{lemma}{Lemma}
\newtheorem{corollary}{Corollary}
\newtheorem{proposition}{Proposition}
\newtheorem{definition}{Definition}
\newtheorem{problem}{Problem}

%===== math macros =====

\def\bm{\boldmath}
%\def\defeq{\stackrel{\textrm{\tiny{def}}}{=}}
\def\defeq{:=}
\def\eps{\epsilon}
\def\fr{\frac}
\def\-{\mbox{-}}
\def\inte{\mathbb{N}}
\def\ovline{\overline}
\def\real{\mathbb{R}}

\def\lc{\lceil}
\def\lf{\lfloor}
\def\rc{\rceil}
\def\rf{\rfloor}

\def\nn{\nonumber}

\def\Pr{\mathbf{Pr}}
\def\expt{\mathbf{E}}
\def\var{\mathbf{var}}

\def\*{\star}

\DeclareMathOperator*{\argmin}{arg\,min}
\DeclareMathOperator*{\polylg}{polylg}
\DeclareMathOperator*{\polylog}{polylog}
\DeclareMathOperator*{\intr}{\cap}

%===== misc =====

\def\done{\hspace*{\fill} $\framebox[2mm]{}$}	% end of proof
%\def\done{\hspace*{\fill} $\Box$}	% end of proof

%===== coloring =====

\newcommand{\red}[1]{\textcolor{red}{#1}}


\def\extraspacing{\vspace{5mm} \noindent}

\def\best{\mathit{best}}
\def\size{\mathit{size}}

\newboolean{solver}\setboolean{solver}{true}
%\newboolean{solver}\setboolean{solver}{false}
\ifthenelse{\boolean{solver}}{\includecomment{sol}}{\excludecomment{sol}}

\pagenumbering{gobble}

\begin{document}

.

\vspace{50mm}

\begin{center}
    \uline{SC3020-CE4301-CZ4031: Quiz 2}
\end{center}

\vspace{20mm}
\begin{center}
    Name:
    \uline{\hspace{30mm}}
    \hspace{10mm}
    Student ID:
    \uline{\hspace{30mm}}
\end{center}

\pagebreak

\extraspacing {\bf Problem 1 (60 marks).} Consider three relations: $R_1(A, B)$ --- that is, a relation with attributes $A$ and $B$ --- $R_2(A, C)$, and $R_3(A, D)$, where $A$, $B$, and $C$ are all integer  attributes. We are given the following query

\mytab{
    \ttt{select} * \ttt{from} $R_1, R_2, R_3$
    \ttt{where} $R_1.A = R_2.A$ \ttt{and} $R_2.A = R_3.A$ \ttt{and} $R_1.A \ge 100$ \ttt{and} $R_3.D \ge 100$;
}

\noindent and the following facts:

\myitems{
    \item Fact 1: Each disk block stores 20 integer values. \vspace{-1mm}
    \item Fact 2: $R_1, R_2$, and $R_3$ have $10^6$, $10^8$, and $10^6$ tuples, respectively. They occupy $10^5$, $10^7$, and $10^5$ disk blocks, respectively. \vspace{-1mm}
    \item Fact 3: For each tuple $t_1 \in R_1$, there are 100 tuples in $t_2 \in R_2$ satisfying $t_1.A = t_2.A$. \vspace{-1mm}
    \item Fact 4: The attribute $A$ is the primary key of $R_1$ and $R_3$. \vspace{-1mm}
    \item Fact 5: There is a clustered B-tree on $R_1$ indexing the attribute $A$. The B-tree has 3 levels. \vspace{-1mm}
    \item Fact 6: 10\% of the tuples in $R_1$ satisfy $A \ge 100$ and 1\% of the tuples in $R_3$ satisfy $D \ge 100$. \vspace{-1mm}
    \item Fact 7: The memory has $M = 900$ blocks. \vspace{-1mm}
}

\noindent Suppose that we process the query using the plan below:
\myenums{
    \item Use the B-tree to find and materialize $R_4 = \{\text{tuple } t_1 \in R_1 \mid t_1.A \ge 100\}$. \vspace{-1mm}

    \item  Compute $
        R_5 = R_4 \bowtie R_2 $
    using hash-join (HJ) and materialize it. \vspace{-1mm}

    \item Scan $R_3$ to find and materialize $R_6 = \{\text{tuple } t_3 \in R_3 \mid t_3.D \ge 100\}$. \vspace{-1mm}

    \item Compute $R_5 \bowtie R_6$ using sort join (SJ). \vspace{-1mm}
}
\noindent Provide an argument to show that the plan performs no more than {\bf $42$ million} I/Os, assuming that the query result is displayed on screen. Whenever needed, you can make the good hashing assumption for HJ.

\begin{sol}
    \extraspacing {\bf Solution.} By Fact 6, $R_4$ has $10^5$ tuples and fits in $10^4$ blocks. Using the B-tree, Step 1 can be done in $3 + 10^4 \text{ (for reading)} + 10^4 \text{ (for writing)} = 20003$ I/Os. As each tuple of $R_5$ has 3 integers, a block stores $\lf 20/3 \rf = 6$ tuples of $R_5$. By Fact 3, $R_5$ has $10^5 \times 100 = 10^7$ tuples and fits in $\lc 10^7 / 6 \rc = 1,666,667$ I/Os. Hence, Step 2 can be done in $3 \times (10000 + 10^7) + 1,666,667 = 31,676,667$ I/Os. By Fact 6, Step 3 takes $10^5 \text{ (for reading)} + 10^3 \text{ (for writing)} = 101,000$ blocks. For Step 4, the no-skew assumption holds because $A$ is the primary key of $R_3$. Thus, the step can be implemented in $5 \times (1,666,667 + 1000) = 8,338,335$ I/Os. The total I/O cost is therefore $41,701,672$ I/Os.
\end{sol}


\pagebreak
.

\vspace{100mm}

\extraspacing {\bf Problem 2 (40 marks).} Find a plan that can process the query in at most {\bf $21$ million} I/Os, assuming that the query result is displayed on screen. Whenever needed, you can make the good hashing assumption for HJ.

\vgap

You get 5 marks if your plan performs at most $37$ million I/Os.

\begin{sol}
    \extraspacing {\bf Solution.} Use the B-tree to find and materialize $R_4 = \{\text{tuple } t_1 \in R_1 \mid t_1.A \ge 100\}$ in 20003 I/Os. Scan $R_3$ to find and materialize $R_5 = \{\text{tuple } t_3 \in R_3 \mid t_3.D \ge 100\}$ in 101,000 I/Os. Use HJ to compute and materialize $R_6 = R_4 \bowtie R_5$ in $3 \times (10,000 + 1,000) = 33,000$ I/Os. Each block stores 6 tuples of $R_6$. As $R_6$ has at most $|R_5| = 10^4$ tuples ($A$ is the primary key of $R_6$), it fits in $\lc 10^4 / 6 \rc = 1667$ blocks. Use BNL to compute $R_2 \bowtie R_6$ in $1667 + \lc 1667 / 899 \rc \cdot 10^7 = 20,001,667$ blocks. The total I/O cost is therefore $20,155,670$.
\end{sol}

\end{document}
