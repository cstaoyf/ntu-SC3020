%\documentclass{...}

%modification history
% 9 jul 2011
% 29 nov 2011

\documentclass[11pt, letterpaper]{article}
\usepackage[margin=1in]{geometry}

% \documentclass[11pt, letterpaper]{article}
% 
% % ----- margins -----
% 
% \topmargin -1.5cm         % read Lamport p.163
% \oddsidemargin -0.04cm    % read Lamport p.163
% \evensidemargin -0.04cm   % same as oddsidemargin but for left-hand pages
% 
% % ----- texts -----
% 
% \textwidth 16.59cm
% \textheight 21.94cm
% 
% % ----- indendts and spacing -----
% 
% \parskip 0pt            	% spacing between paragraphs
% %\renewcommand{\baselinestretch}{1.5}	% uncomment for 1.5 spacing
% 
% \parindent 7mm		      % leading space for paragraphs between lines
% 
% % ----- page # -----
% 
% %\pagestyle{empty}         % uncomment if don't want page numbers





\usepackage{amsfonts, amsmath, amssymb, amsthm}
\usepackage{comment}
\usepackage{graphicx}
\usepackage{ifthen}
\usepackage{latexsym}
%\usepackage{times}
\usepackage[normalem]{ulem}

%modification history
% 2012
% may 3
% aug 11

%=== use the these packages if they are not already in use ===
%\usepackage{amsfonts, amsmath, amssymb, amsthm}
%=============================================================

%===== fonts =====
\def\ttt{\texttt}
%===== spacing =====

\def\extraspacing{\vspace{5mm} \noindent}
\def\figcapup{\vspace{-1mm}}
\def\figcapdown{\vspace{-0mm}}
\def\hgap{\textrm{\hspace{1mm}}}
\def\thmvgap{\vspace{0mm}}
\def\vgap{\vspace{2mm}}


%===== tabbing =====

\def\tab{\hspace{3mm}}
\def\tabpos{\hspace{4mm} \= \hspace{4mm} \= \hspace{4mm} \= \hspace{4mm} \=
\hspace{4mm} \= \hspace{4mm} \= \hspace{4mm} \= \hspace{4mm} \= \hspace{4mm}
\kill}

%===== blocks =====

\newtheorem{theorem}{Theorem}
\newtheorem{lemma}{Lemma}
\newtheorem{corollary}{Corollary}
\newtheorem{proposition}{Proposition}
\newtheorem{definition}{Definition}
\newtheorem{problem}{Problem}

%===== math macros =====

\def\bm{\boldmath}
%\def\defeq{\stackrel{\textrm{\tiny{def}}}{=}}
\def\defeq{:=}
\def\eps{\epsilon}
\def\fr{\frac}
\def\-{\mbox{-}}
\def\inte{\mathbb{N}}
\def\ovline{\overline}
\def\real{\mathbb{R}}

\def\lc{\lceil}
\def\lf{\lfloor}
\def\rc{\rceil}
\def\rf{\rfloor}

\def\nn{\nonumber}

\def\Pr{\mathbf{Pr}}
\def\expt{\mathbf{E}}
\def\var{\mathbf{var}}

\def\*{\star}

\DeclareMathOperator*{\argmin}{arg\,min}
\DeclareMathOperator*{\polylg}{polylg}
\DeclareMathOperator*{\polylog}{polylog}
\DeclareMathOperator*{\intr}{\cap}

%===== misc =====

\def\done{\hspace*{\fill} $\framebox[2mm]{}$}	% end of proof
%\def\done{\hspace*{\fill} $\Box$}	% end of proof

%===== coloring =====

\newcommand{\red}[1]{\textcolor{red}{#1}}


\def\extraspacing{\vspace{5mm} \noindent}

\def\best{\mathit{best}}
\def\size{\mathit{size}}

\newboolean{solver}\setboolean{solver}{true}
%\newboolean{solver}\setboolean{solver}{false}
\ifthenelse{\boolean{solver}}{\includecomment{sol}}{\excludecomment{sol}}

\pagenumbering{gobble}

\begin{document}

.

\vspace{50mm}

\begin{center}
    \uline{SC3020-CE4301-CZ4031: Quiz 3}
\end{center}

\vspace{20mm}
\begin{center}
    Name:
    \uline{\hspace{30mm}}
    \hspace{10mm}
    Student ID:
    \uline{\hspace{30mm}}
\end{center}

\pagebreak

\extraspacing {\bf Problem 1 (40 marks).} Consider the following three transactions:

\begin{center}
    \begin{tabular}{l|l|l}
        $T_1$ & $T_2$ & $T_3$ \\
        \hline
        U-LOCK(X) & S-LOCK(Y) & U-LOCK(Y) \\
        READ(X) & READ(Y) & READ(Y) \\
        UPGRADE(X) & X-LOCK(X) & UPGRADE(Y) \\
        WRITE(X) & READ(X) & WRITE(Y) \\
        COMMIT & WRITE(X) & COMMIT \\
        & COMMIT &
    \end{tabular}
\end{center}

\noindent Suppose that we use the strict two-phase locking (2PL) protocol to execute the transactions in a round-robin manner --- namely, each cycle first processes an instruction from $T_1$, then one from $T_2$, and finally one from $T_3$. Answer the following questions.
\myitems{
    \item [(a)] What is the schedule produced? \vspace{-2mm}
    \item [(b)] Show the precedence graph of the schedule in (a). \vspace{-2mm}
    \item [(c)] Give all the serial orders that are equivalent to the schedule in (a).
}

\begin{sol}
    \extraspacing {\bf Solution.} (a)

    \begin{center}
    \begin{tabular}{c|l|l}
        no & operation & remark \\
        \hline
        1 & $T_1$: U-LOCK(X) & Granted (no locks on $X$) \\
        2 & $T_2$: S-LOCK(Y) & Granted (no locks on $Y$) \\
        3 & $T_3$: U-LOCK(Y) & Granted (compatible with $T_2$’s S-lock) \\
        4 & $T_1$: READ(X) & \\
        5 & $T_2$: READ(Y) & \\
        6 & $T_3$: READ(Y) & \\
        7 & $T_1$: UPGRADE(X) & Upgraded successfully to X-lock (no conflict) \\
        8 & $T_2$: X-LOCK(X) & Blocked (conflicts with $T_1$’s X-lock) \\
        9 & $T_3$: UPGRADE(Y) & Blocked (conflicts with $T_2$’s S-lock on $Y$) \\
        10 & $T_1$: WRITE(X) & \\
        11 & $T_1$: COMMIT & Releases all locks on $X$ \\
        12 & $T_2$: X-LOCK(X) & Granted (after $T_1$’s commit) \\
        13 & $T_2$: READ(X) & \\
        14 & $T_2$: WRITE(X) & \\
        15 & $T_2$: COMMIT & Releases locks on $X$ and $Y$ \\
        16 & $T_3$: UPGRADE(Y) & Now succeeds (no other locks on $Y$) \\
        17 & $T_3$: WRITE(Y) & \\
        18 & $T_3$: COMMIT &
    \end{tabular}
    \end{center}

    (b) $T_1 \rightarrow T_2 \rightarrow T_3$

    (c) There is only one equivalent serial order: $T_1, T_2, T_3$.
\end{sol}

\pagebreak
.

\vspace{100mm}

\extraspacing {\bf Problem 2 (40 marks).} Suppose that the database attempts to execute the following instructions in the order shown using the MVCC protocol.

\begin{center}
\begin{tabular}{c|l|l|l}
    & \(T_{1}\) (ID 101) & \(T_{2}\) (ID 102) & \(T_{3}\) (ID 103)  \\
    \hline
    1 & \(\text{READ}(t)\) & & \\
    2 & \(\text{LOCK}(t)\) & & \\
    3 & \(\text{WRITE}(t, 12)\) & & \\
    4 & & \(\text{LOCK}(t)\) & \\
    5 & & & \(\text{READ}(t)\) \\
    6 & & \(\text{WRITE}(t, 20)\) & \\
    7 & & \(\text{ABORT}\) & \\
    8 & READ($t$) & \\
    9 & \(\text{COMMIT}\) & & \\
    10 & & & \(\text{LOCK}(t)\) \\
    11 & & & \(\text{READ}(t)\) \\
    12 & & & \(\text{WRITE}(t, 25)\) \\
    13 & & & \(\text{COMMIT}\)
\end{tabular}
\end{center}

\noindent Before the execution, the history of $t$ is $(10, 100, \text{NULL})$, and the transaction with ID 100 has committed. Answer the following questions:
\myitems{
    \item [(a)] Show the schedule produced. \vspace{-2mm}
    \item [(b)] Show the values fetched by the READ operations at Instructions 1, 5, 8, and 11. \vspace{-2mm}
    \item [(c)] What is the history of $t$ at the end of the execution?
}

\begin{sol}
\extraspacing {\bf Solution.} (a)

\begin{center}
\begin{tabular}{c|l|l}
no & operation & remark \\
\hline
1 & $T_1:$ READ($t$) \\
2 & $T_1:$ LOCK($t$) \\
3 & $T_1:$ WRITE($t, 12$) \\
4 & $T_2:$ LOCK($t$) & blocked \\
5 & $T_3:$ READ($t$) & \\
6 & $T_1:$ READ($t$) \\
7 & $T_1:$ COMMIT \\
8 & $T_2:$ LOCK($t$) & \\
9 & $T_2:$ WRITE($t$, 20) \\
10 & $T_2:$ ABORT \\
11 & $T_3:$ LOCK($t$) & \\
12 & $T_3:$ READ($t$) \\
13 & $T_3:$ WRITE($t, 25$) \\
14 & $T_3:$ COMMIT
\end{tabular}
\end{center}

(b) 10, 10, 12, 12.

(c) (10, 100, 101), (12, 101, 103), (25, 103, NULL).

\end{sol}
\pagebreak
.

\pagebreak

\extraspacing {\bf Problem 3 (20 marks).} Suppose that the database system sees the following log after a reboot.

\begin{center}
\begin{tabular}{l}
$\langle \text{START } T_1 \rangle$ \\
$\langle \text{START } T_2 \rangle$ \\
$\langle T_1, P, 40, 90 \rangle$ \\
$\langle T_2, Q, 25, 45 \rangle$ \\
$\langle \text{START } T_3 \rangle$ \\
$\langle T_3, R, 300, 280 \rangle$ \\
$\langle \text{COMMIT } T_2 \rangle$ \\
$\langle T_1, Q, 45, 70 \rangle$ \\
$\langle \text{COMMIT } T_1 \rangle$ \\
$\langle T_3, P, 90, 130 \rangle$
\end{tabular}
\end{center}

\noindent Describe the steps taken by the recovery procedure.

\begin{sol}
\extraspacing {\bf Solution.} $T_1$ and $T_2$ had committed before the crash while $T_3$ had not. The redo phase processes the log forward and performs three disk modifications (in this order):
\myitems{
    \item set $P$ to 90;
    \item set $Q$ to 45;
    \item set $Q$ to 75,
}
The undo phase processes log backward and performs two disk modifications (in this order):
\myitems{
    \item set $P$ to 90;
    \item set $R$ to 300.
}
\end{sol}

% \pagebreak
% .
% \vspace{100mm}


\end{document}
